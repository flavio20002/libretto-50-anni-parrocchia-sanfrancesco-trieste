\begin{flushright}
\needspace{3\baselineskip}
Trieste, 3 ottobre 1976
\end{flushright}
Miei Fratelli,\par
la domenica 15 Ottobre 1967 mi presentavo a voi assumendo il servizio pastorale di questa comunità parrocchiale, oggi celebro per voi e con voi la S. Messa passando ad altre mani il mio mandato.
Sono passati 9 anni: possono essere considerati molti o pochi a seconda del punto di vista dal quale ci si colloca, è un normale avvicendamento, sia pure non richiesto, sono stati comunque vissuti 
questi nove anni, mi sembra di poter dire con intensità di operosità, di conoscenza reciproca, di stima, di crescita nella comune vocazione alla risposta della parola di Dio.
Mi sia consentito, sia pure brevemente, di ricordare qualche tratto del cammino svolto insieme in questi anni:
in continuità di opere con il mio predecessore mi sono anzitutto dedicato a rendere sempre più accogliente la chiesa, come luogo di riunione e convegno di preghiera: 
ecco i banchi nuovi offerti in memoria dei vostri defunti, le suppellettili e arredi per una nobile e dignitosa celebrazione delle sacre funzioni, il riscaldamento della chiesa esteso anche al 
Franciscanum e Oratorio, il rimaneggiamento della pavimentazione, la ristrutturazione dell’impianto di illuminazione e i sei grandi pendenti, l’impianto di amplificazione e finalmente
il grande, lungo, faticoso restauro della superficie muraria interna con la demolizione degli intonaci e la pulitura delle superfici dei conci di pietra arenaria e successiva fugatura così
da portare a vista tutta la superficie interna e dare un volto nuovo, vibrante e austero, a tutta l'architettura e significare così, almeno per immagine, la vera struttura della chiesa, corpo
mistico di Cristo, costituita da pietre vive.
La non indifferente somma di denaro per far fronte e sostenere le spese per detti lavori è stata, reperibile grazie alla alla vostra generosità e ad una accorta amministrazione senza lasciare
attualmente alcuna passività.
Ma la mia cura maggiore si è rivolta non tanto alle opere da fare, sia pure utili o necessarie, quanto alle persone come componenti vive della comunità cristiana dirigendo in particolare 
la mia azione verso il ministero di\\
a) dispensatore della parola intesa come evangelizzazione, infatti quante prediche avete sentito... e le istruzioni nelle diverse Associazioni (Azione Cattolica, Terz'Ordine, Milizia, 
S. Vincenzo, giornate particolari con tridui, novene ecc.... assemblee, incontri di mamme e papà ... e la preparazione catechistica ai bambini per i sacramenti della iniziazione cristiana e 
agli adulti per vivificare il senso di consapevolezza e responsabilità dei loro doveri cristiani nei confronti dei figli....da questa maturazione ecco l'aspetto di\\
b) dispensatore dei sacramenti: come veri incontri con la grazia di Dio che fa rinascere nel battesimo, corrobora nella Cresima, nutre nell'Eucarestia, perdona nella riconciliazione, 
consacra l'amore nel matrimonio, risana e apre a speranza nell'unzione degli infermi.\\Lo sforzo ancora e l'impegno all'interno di questa famiglia di farmi\\
c) dispensatore di carità, nell'accoglienza e nel servizio tutte le persone nelle visite alle famiglie; gli ammalati e i vecchi in modo particolare, nella preghiera fatta insieme qui in chiesa,
nelle vostre abitazioni e anche nei vostri cortili come non ricordare il maggio dell'anno scorso, l'anno santo, e l'attenzione ecumenica fatta di conoscenza 
e rispettosa deferenza verso i Fratelli separati...\\
Intenzione esplicita e come forza motrice in questi anni mi è stata quella di creare tra noi tutti un clima di famiglia, rendere la chiesa veramente la casa di tutti, la casa comune, dove ognuno si
sentisse a suo agio, si sentisse bene, e vi potesse aprire serenamente e fiduciosamente il suo animo e ...
quindi le Associazioni Cattoliche sì, ma volutamente non ho prestato attenzione a gruppi o gruppuscoli che avessero potuto essere elementi di separazione e di isolamento nel tessuto della
comunità. Con particolare predilezione ho curato i chierichetti, sempre bravi e numerosi alle celebrazioni; il settore giovani è stato quello che per diverse cause, 
sia logistiche di spazi e ambienti che per mancanze di personale ha più sofferto.
Ai momento quindi di accomiatarmi, ringrazio tutti voi per le preziose e attente collaborazione prestate, a sviluppare questi miei intendimenti che il Signore Dio giudicherà nella 
loro realizzazione.
Essendo la parrocchia una parte della chiesa locale ossia della Diocesi rivolgo il mio deferente pensiero al Vescovo che ora la governa e in particolare a Mons. Santin con il quale
ho maggiormente intrattenuto relazioni di collaborazione e di rispettosa e filiale vicinanza.
È per me doveroso ricordare e accomunare nel ringraziamento tutti i confratelli che hanno con me condiviso in questi anni (anche se ora si trovano lontano da noi) 
la responsabilità e le fatiche dell'apostolato nei diversi settori pastorali e qui rappresentati nella concelebrazione da P. Luigi... 
Il mio successore: non spetta a me presentarlo, lo farà ufficialmente il Vescovo domani durante la Messa Vespertina; mi sia consentito solamente di dire che egli viene in mezzo a voi ricco di
esperienza pastorale avendo già retto per undici anni la parrocchia del Tempio Votivo a Verona, e secondariamente, un fatto personale: il P. Innocenzo era allora sacerdote novello a 
Camposampiero e per quei miei primi anni di Seminario è stato il mio Direttore Spirituale...una certa continuità...
\begin{flushright}
\includegraphics[scale=0.16]{immagini/firmaBaldassa.jpg}
\end{flushright}
%\newpage