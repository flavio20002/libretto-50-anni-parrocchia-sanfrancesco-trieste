\chapter*{Saluto dell’Arcivescovo\\ di Trieste}
\addcontentsline{toc}{chapter}{Saluto dell’Arcivescovo di Trieste}
Il 3 ottobre 1965, il mio amato predecessore S. E. Mons. Antonio Santin, con un decreto  
fondazionale, provvedeva ad erigere, nella Diocesi di Trieste, la parrocchia dedicata a San 
Francesco, affidandola alla cura pastorale dei Frati Conventuali. Il ricordo di quella data è motivo di 
gioiosa riconoscenza al Signore per tutte le grazie che, fino ad oggi, ha riservato alla comunità 
parrocchiale e a tutta la Chiesa di Trieste. La grazia del servizio generoso di tanti padri francescani 
che hanno speso le energie migliori del loro ministero per annunciare il Vangelo e per testimoniare 
la carità verso i poveri; la grazia di tantissimi bambini e bambine, di giovani che in parrocchia 
hanno imparato ad amare Dio e ad essere cittadini esemplari; la grazia di tante famiglie che nella 
comunità hanno rafforzato il loro amore e affrontato tante difficoltà; la grazia di tanti poveri, di 
tante persone sole e anziane, che nei fratelli e nelle sorelle della comunità di  san Francesco  hanno 
trovato consolazione e sostegno; la grazia di tanti collaboratori pastorali – catechisti, volontari, 
operatori della caritas… - che, nell’arco di questi cinquant’anni, hanno impreziosito la comunità 
con la testimonianza genuina delle loro opere buone; la grazia della Parola - quella di Gesù e quella 
che è Gesù - che ha convertito e rigenerato innumerevoli anime; la grazia dei sacramenti, soprattutto 
quelli dell’Eucaristia, del Battesimo e della Confessione che sono stati il nutrimento indispensabile 
per dare vita e identità autentiche alla comunità cristiana; la grazia della preghiera e quella della 
vocazione alla santità che Dio rivolge a tutti.
Cinquant’anni di vita cristiana vissuta insieme sotto la protezione di San Francesco; 
cinquant’anni che hanno registrato il miracolo quotidiano dell’amore misericordioso di Dio. Ora, 
doverosamente, la memoria deve aprire i cuori e le menti al futuro affinché la comunità si 
predisponga a viverlo - nella fede, nella speranza e nella carità cristiane - con la stessa fedeltà e il 
medesimo entusiasmo.
Nell’affidare la parrocchia di San Francesco alla materna protezione di Maria, Madre della 
Chiesa, colgo l’occasione per benedire tutti e di cuore.
\begin{flushright}
+Giampaolo Crepaldi\par
Arcivescovo – Vescovo di Trieste
\end{flushright}
