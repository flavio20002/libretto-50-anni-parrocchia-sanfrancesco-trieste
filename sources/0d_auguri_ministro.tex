\chapter*{Augurio del Ministro Provinciale\\ OFM Conv.}
\addcontentsline{toc}{chapter}{Augurio del Ministro Provinciale OFM Conv.}
\begin{center}
\bfseries
	Per i 50 anni della nostra Parrocchia\par San Francesco in Trieste
\end{center}
``Ti adoriamo, Signore nostro Gesù Cristo, qui e anche in tutte le chiese che sono nel mondo 
intero e ti benediciamo, perché con la tua santa croce hai redento il mondo” (Fonti francescane 
111). Mi affiora alla mente questa preghiera del Serafico Padre nella lieta ricorrenza dei 50 anni 
della Parrocchia di Trieste intitolata proprio a lui ed affidata dalla Diocesi all’animazione della 
nostra Provincia religiosa di Frati Minori Conventuali. È una preghiera che Francesco trovò ed 
elaborò sino a farla sua per poi consegnarla ai suoi frati e a quanti incontrava; un testo che gli 
faceva sobbalzare il cuore. Se ne intendeva di chiese il giovane Francesco: era nata tra queste, fra le 
più derelitte, la sua vocazione, giù nella piana di Assisi, ai margini della città, tra chiese 
abbandonate e corpi trascurati di lebbrosi. Entrambe queste realtà lo videro “restauratore” con 
mattoni e con lo sguardo, la vicinanza fattiva di chi dà dignità a chi l’aveva smarrita. Ognuna di 
queste “due” chiese, quella cadente di San Damiano e quella sofferente dei fratelli lebbrosi, aveva 
bisogno l’una dell’altra e rimandava al proprio tesoro: il Signore Gesù presente nella chiesa di 
mattoni e nei fratelli poveri, messi ai margini, scartati dai più. Iniziò così l’avventura che potremmo 
definire ``parrocchiale'' di Francesco che sino a quel momento della sua vita, si era concentrato 
unicamente su di sé, sui suoi sogni di autorealizzazione. Parafrasando un po' un testo noto: ``Dolce è 
sentire - è la scoperta di Francesco! - che non sono più solo, ma che son parte di una immensa vita, 
di una comunità di fratelli e sorelle, dono del Signore, del suo infinito amore…''
\bigbreak
``Francesco, va’, ripara la mia casa che, come vedi, è tutta in rovina'' (Fonti Francescane 
593). Cominciò così, con l’adesione semplice ed immediata all’invito del Crocifisso di San 
Damiano che gli aveva parlato trafiggendo di dolcezza il suo cuore, la vocazione di Francesco. 
Anche come manovale e muratore. Da allora, sino alla piena “conformitas”- conformità con il 
Crocifisso nel segno delle stigmate sul monte della Verna, quasi al finire dei suoi giorni, l’amore 
per il suo Signore fattosi uomo per noi, fu il centro del cuore di frate Francesco, dei suoi affetti, 
della sua predicazione. Meditava sempre, dice il suo primo biografo, fra Tommaso da Celano, 
“l’umiltà dell’incarnazione e la carità della passione” di Gesù, sino a piangere di commozione e di 
sofferenza quando constatava che “l’Amore non è amato”. Volle assomigliargli in tutto: con l’abito 
di sacco fatto a forma di croce, scegliendo il segno del Tau che riproduce la croce e, soprattutto, 
volendo per sé e i suoi frati una via di povertà e condivisione da vivere in letizia.

Fu solo più avanti che frate Francesco comprese che l’invito a “riparare la chiesa”
riguardava non solo gli edifici cadenti, ma l’edificio vivente della Chiesa. A dirla tutta, quella  del 
suo tempo era una Chiesa non molto edificante, anzi, bisognosa  di riforma, di un ritorno autentico 
al Vangelo, alla sua freschezza. Le chiese “poverelle” ed abbandonate incontrate da Francesco fuori 
Assisi erano come lo specchio della decadenza della Chiesa che attendeva  una vera e propria 
rifondazione. Frate Francesco e i suoi frati - i frati minori: un nome, un programma  - vi riuscirono: 
fu un “restauro” perfetto perché avvenne non “contro” la Chiesa, ma in obbedienza ad essa, 
accompagnandola a farle intravedere la bellezza e possibilità di vivere il Vangelo della fraternità e 
della minorità, della povertà e della letizia.

Eppure frate Francesco non si dimenticò i primi restauri che lo videro manovale e tuttofare 
nelle chiese povere e malandate: quando ne trovava in disordine, raccontano le Fonti Francescane, 
lui stesso cercava una scopa per spazzarle, si occupava del decoro delle tovaglie, delle suppellettili 
d’altare, coinvolgendo in questo progetto Chiara e le sue Povere Dame (in seguito denominate 
Clarisse). E così voleva facessero i frati. Perché? A motivo che qui - proprio in questa chiesa, qui in 
questa comunità ove ora mi trovo! - e in tutte le chiese che sono disseminate nel mondo, il Signore 
Gesù va adorato perché con la sua santa croce ha amato me e ogni persona di questa terra, ha 
salvato il mondo, gli uomini e le donne di tutti i tempi, offrendo tutto se stesso. Perché qui, in 
questa chiesa, qui in questa comunità parrocchiale, possiamo fare esperienza di lui, perché egli qui 
mi e ci raggiunge con il suo amore. Qui il Signore ci ha voluti per crescere insieme come famiglia 
parrocchiale.
%\bigbreak

Nel celebrare i 50 anni della nostra Parrocchia di San Francesco in Trieste, auguro ai
confratelli e fedeli di poter vivere appieno gli atteggiamenti del cuore del Poverello di Assisi che 
della Comunità è il Patrono. La gioia di entrare, come singoli e come Comunità riunita in 
assemblea, nella bella chiesa di Via Giulia, per incontrare-lasciarsi incontrare ed adorare qui il 
Signore Gesù, per lasciarsi invadere dal suo amore che ci raggiunge nei sacramenti e nella preghiera 
assieme elevata alla Trinità santa; la letizia di scoprirsi una Fraternità parrocchiale radunata nel suo 
nome e da lui continuamente riconciliata; la responsabilità di “riparare la Chiesa”, anzitutto volendo 
bene ai frati, e alla Chiesa che ci è madre, aiutandoli con il proprio contributo di idee e la propria 
testimonianza nella ricerca del vero bene della Comunità; l’uscita in città, negli ambienti di vita ed 
anche in periferia, - come ci esorta Papa Francesco - per “fare misericordia” portando il Vangelo 
della fraternità a chi più soffre, a chi si sente messo ai margini.
La Parrocchia tutta - frati, suore, fedeli laici - sarà allora una Fraternità francescana che, meglio che 
può, con passione ed intelligenza,  segue e serve il Signore Gesù in letizia e minorità.
%\bigbreak

Infine, per un compleanno così significativo, 50 anni!, corrono d’obbligo tre movimenti.
\begin{itemize}
\item Ringraziare per la strada sin qui compiuta. Ci sono storie, volti, esperienze di frati, 
parrocchiani che il cuore grande della Comunità contiene e sono indispensabili per tener viva 
l’identità e l’appartenenza. Una memoria cui ritornare, come ad una buona sorgente, con 
gratitudine, senza però troppo indulgere nella nostalgia ed affidando al Signore quanto ancora 
chiede di essere purificato e riconciliato pienamente.
\item Vivere il presente con fiducia e passione. La memoria grata del passato spinge, in ascolto 
attento di ciò che oggi lo Spirito Santo dice alla Chiesa, ad attuare in modo sempre più convinto 
l’adesione al Vangelo, calandolo nelle sfide e potenzialità dell’oggi. Con fede e passione.
\item Guardare il futuro con speranza. Lo può fare una comunità che coltivando una memoria 
grata si fida del Signore, della sua azione cui “nulla è impossibile”, senza cadere nello 
scoraggiamento e/o nella tentazione dei numeri e dell’efficienza, confidando solo nelle proprie 
forze. Infatti ogni momento della vita personale e comunitaria può essere kairòs ovvero “tempo di 
grazia” e di trasformazione, nel quale il Signore, visitandoci e rincuorandoci, ci fa avanzare per 
sentieri inediti e provvidenziali di pace e di bene.
\end{itemize}
\bigbreak
\noindent Buon compleanno, allora, carissima Comunità parrocchiale di San Francesco! Nella grazia sempre 
sorprendente e sovrabbondante del Signore e con la fraterna intercessione del Patrono, San 
Francesco d’Assisi,  ad multos annos!
\begin{flushright}
fr. Giovanni Voltan\par
ministro provinciale
\end{flushright}
