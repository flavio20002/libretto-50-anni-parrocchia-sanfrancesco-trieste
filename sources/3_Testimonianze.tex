\chapter{I Parroci raccontano}
\label{chap:Testimonianze}

\begin{center}
\textbf{\Large p. FORTUNATO ZORZINI}\\
	\textit{Amministratore parrocchiale dal 1965 al 1967}
\end{center}
%\textbf{Vicari: p. Stanislao Sgarbossa e p. Salvatore Stagni}
\bigskip
\begin{center}
\textbf{\Large p. OLINDO BALDASSA racconta...}\\
	\textit{parroco dal 1967 al 1976}
\end{center}
%\textbf{Vicari: p. Stanislao Sgarbossa e p. Salvatore Stagni}
\bigbreak
\begin{flushright}
Brescia, 3 ottobre 1976\\
Festa di S. Bonaventura
\end{flushright}
\noindent Nel mio pro-memoria per il Cinquantesimo di ordinazione sacerdotale (2010) “Cinquant’anni di grazia. (spigolature)” riferendomi ai nove anni di attività pastorale nella parrocchia di s. Francesco di Trieste (1967 – 1976) mi ero espresso in questo modo:
\bigbreak 
\enquote{\itshape Quel po' di esperienza pastorale fatta presso la parrocchia dei santi Pietro e Paolo all’Eur (Roma) ancora fresco di ordinazione, e quella acquisita nella direzione dei chierici, sono venute buone quando i superiori mi hanno assegnato al convento di s. Francesco d’Assisi di Trieste con il duplice compito di superiore dei frati e di parroco della parrocchia dal vescovo assegnata ai religiosi conventuali.

Quella triestina è stata una stagione pastoralmente splendida, indimenticabile. Le primizie, infatti, sono sempre deliziose e piacevoli. Dal punto di vista canonico sono stato il primo religioso della comunità a ricoprire il ruolo di parroco della parrocchia triestina di s. Francesco. Se l’esperienza è stata positiva, e non solo per me, lo devo anche alla benevolenza del vescovo, monsignor Antonio Santin, alla cordialità dei sacerdoti diocesani e dei religiosi miei confratelli e, soprattutto, all’amicizia e all’affetto dei parrocchiani, che rimangono come perle preziose incastonate nella collana dei miei giorni. 

Trieste, città di confine e mitteleuropea, è sempre stata sensibile all’approccio ecumenico con altre esperienze religiose vicine e lontane: un percorso nel quale anche la nostra comunità si è efficacemente inserita, con varie iniziative.

Su altro versante, è stato impegnativo ma gratificante, il lavoro di rifacimento dell’interno della chiesa, con la rimozione dell’intonaco da tutta la superficie per liberare la pietra faccia a vista, che ha dato all’edificio un volto nuovo e con il rinnovato impianto di illuminazione, quasi anticipo e preludio dei successivi lavori, pure molto impegnativi, di sistemazione di tutta l’area presbiteriale. 

Del periodo triestino mi piace segnalare alcuni avvenimenti particolari, ancora vivi nel ricordo, come il viaggio ecumenico della diocesi a Istanbul, guidato da monsignor Santin e don Eugenio Ravignani, culminato nell’incontro con il Patriarca ortodosso Athenagoras I e il mio primo pellegrinaggio in Terra Santa, nel 1973, offertomi dai parrocchiani.}
\bigbreak

Vengo ora a ripetere e rinnovare gli stessi sentimenti anche in questa circostanza, aggiungendovi qualche nota. 

Il decreto canonico di erezione a parrocchia veniva a confermare quanto già da circa venti anni si svolgeva in quella zona dello Scoglietto – Guardiella con il ritorno dei Frati Minori Conventuali. 

Non è stato perciò difficile inserirmi in quel contesto così bene avviato e consolidato con tutte le attività proprie di una comunità cristiana già ben collocata nel territorio diocesano. Preziosa la collaborazione con le Suore Dimesse. Cura e premura particolari del parroco si sono subito rivolte alla conoscenza diretta e personale nella fatica e nella gioia della benedizione e incontri familiari… 

Siamo agli inizi della riforma liturgica e della ripresa del cammino ecumenico: fare della parrocchia una comunità alimentata dalla liturgia feriale e festiva, curando specialmente le celebrazioni comunitarie dei battesimi, anniversari di matrimonio, prime comunioni, cresime, prime messe dei sacerdoti novelli dell’Istituto Teologico di Padova, fioretto di Maggio all’altare della Madonna con uno stuolo di irrequieti chierichetti o nei vari cortili in mezzo alle abitazioni, il pellegrinaggio annuale al santuario di Monte Grisa… e altri incontri di spiritualità e condivisione…. Un capitolo particolare merita la cura e attenzione all’oratorio per i ragazzi e i giovani. 

I sentimenti che hanno riempito la mia stagione pastorale triestina sono stati di dialogo, rispetto, collaborazione, fraternità e sincera amicizia, largamente ricambiati. Di tutto questo ancora ringrazio il Signore e auguro alla Comunità religiosa e parrocchiale di proseguire con lena sulla strada della nuova evangelizzazione secondo le incalzanti indicazioni di Papa Francesco. 
\begin{flushright}
\textit{P. Olindo M. Baldassa}
\end{flushright}
\bigbreak
\textit{Riteniamo utile, a completamento di quanto ci ha raccontato p. Olindo, riportare il suo saluto alla parrocchia, nel giorno della sua partenza (3 ottobre 1976):}
\bigbreak
\begin{flushright}
\needspace{3\baselineskip}
Trieste, 3 ottobre 1976
\end{flushright}
Miei Fratelli,\par
la domenica 15 Ottobre 1967 mi presentavo a voi assumendo il servizio pastorale di questa comunità parrocchiale, oggi celebro per voi e con voi la S. Messa passando ad altre mani il mio mandato.
Sono passati 9 anni: possono essere considerati molti o pochi a seconda del punto di vista dal quale ci si colloca, è un normale avvicendamento, sia pure non richiesto, sono stati comunque vissuti 
questi nove anni, mi sembra di poter dire con intensità di operosità, di conoscenza reciproca, di stima, di crescita nella comune vocazione alla risposta della parola di Dio.
Mi sia consentito, sia pure brevemente, di ricordare qualche tratto del cammino svolto insieme in questi anni:
in continuità di opere con il mio predecessore mi sono anzitutto dedicato a rendere sempre più accogliente la chiesa, come luogo di riunione e convegno di preghiera: 
ecco i banchi nuovi offerti in memoria dei vostri defunti, le suppellettili e arredi per una nobile e dignitosa celebrazione delle sacre funzioni, il riscaldamento della chiesa esteso anche al 
Franciscanum e Oratorio, il rimaneggiamento della pavimentazione, la ristrutturazione dell’impianto di illuminazione e i sei grandi pendenti, l’impianto di amplificazione e finalmente
il grande, lungo, faticoso restauro della superficie muraria interna con la demolizione degli intonaci e la pulitura delle superfici dei conci di pietra arenaria e successiva fugatura così
da portare a vista tutta la superficie interna e dare un volto nuovo, vibrante e austero, a tutta l'architettura e significare così, almeno per immagine, la vera struttura della chiesa, corpo
mistico di Cristo, costituita da pietre vive.
La non indifferente somma di denaro per far fronte e sostenere le spese per detti lavori è stata, reperibile grazie alla alla vostra generosità e ad una accorta amministrazione senza lasciare
attualmente alcuna passività.
Ma la mia cura maggiore si è rivolta non tanto alle opere da fare, sia pure utili o necessarie, quanto alle persone come componenti vive della comunità cristiana dirigendo in particolare 
la mia azione verso il ministero di\\
a) dispensatore della parola intesa come evangelizzazione, infatti quante prediche avete sentito... e le istruzioni nelle diverse Associazioni (Azione Cattolica, Terz'Ordine, Milizia, 
S. Vincenzo, giornate particolari con tridui, novene ecc.... assemblee, incontri di mamme e papà ... e la preparazione catechistica ai bambini per i sacramenti della iniziazione cristiana e 
agli adulti per vivificare il senso di consapevolezza e responsabilità dei loro doveri cristiani nei confronti dei figli....da questa maturazione ecco l'aspetto di\\
b) dispensatore dei sacramenti: come veri incontri con la grazia di Dio che fa rinascere nel battesimo, corrobora nella Cresima, nutre nell'Eucarestia, perdona nella riconciliazione, 
consacra l'amore nel matrimonio, risana e apre a speranza nell'unzione degli infermi.\\Lo sforzo ancora e l'impegno all'interno di questa famiglia di farmi\\
c) dispensatore di carità, nell'accoglienza e nel servizio tutte le persone nelle visite alle famiglie; gli ammalati e i vecchi in modo particolare, nella preghiera fatta insieme qui in chiesa,
nelle vostre abitazioni e anche nei vostri cortili come non ricordare il maggio dell'anno scorso, l'anno santo, e l'attenzione ecumenica fatta di conoscenza 
e rispettosa deferenza verso i Fratelli separati...\\
Intenzione esplicita e come forza motrice in questi anni mi è stata quella di creare tra noi tutti un clima di famiglia, rendere la chiesa veramente la casa di tutti, la casa comune, dove ognuno si
sentisse a suo agio, si sentisse bene, e vi potesse aprire serenamente e fiduciosamente il suo animo e ...
quindi le Associazioni Cattoliche sì, ma volutamente non ho prestato attenzione a gruppi o gruppuscoli che avessero potuto essere elementi di separazione e di isolamento nel tessuto della
comunità. Con particolare predilezione ho curato i chierichetti, sempre bravi e numerosi alle celebrazioni; il settore giovani è stato quello che per diverse cause, 
sia logistiche di spazi e ambienti che per mancanze di personale ha più sofferto.
Ai momento quindi di accomiatarmi, ringrazio tutti voi per le preziose e attente collaborazione prestate, a sviluppare questi miei intendimenti che il Signore Dio giudicherà nella 
loro realizzazione.
Essendo la parrocchia una parte della chiesa locale ossia della Diocesi rivolgo il mio deferente pensiero al Vescovo che ora la governa e in particolare a Mons. Santin con il quale
ho maggiormente intrattenuto relazioni di collaborazione e di rispettosa e filiale vicinanza.
È per me doveroso ricordare e accomunare nel ringraziamento tutti i confratelli che hanno con me condiviso in questi anni (anche se ora si trovano lontano da noi) 
la responsabilità e le fatiche dell'apostolato nei diversi settori pastorali e qui rappresentati nella concelebrazione da P. Luigi... 
Il mio successore: non spetta a me presentarlo, lo farà ufficialmente il Vescovo domani durante la Messa Vespertina; mi sia consentito solamente di dire che egli viene in mezzo a voi ricco di
esperienza pastorale avendo già retto per undici anni la parrocchia del Tempio Votivo a Verona, e secondariamente, un fatto personale: il P. Innocenzo era allora sacerdote novello a 
Camposampiero e per quei miei primi anni di Seminario è stato il mio Direttore Spirituale...una certa continuità...
\begin{flushright}
\includegraphics[scale=0.16]{immagini/firmaBaldassa.jpg}
\end{flushright}
%\newpage
\bigskip
\begin{center}
\textbf{\Large p. INNOCENZO BORDIN}\\
	\textit{parroco dal 1976 al 1985}
\end{center}
\bigskip
\begin{center}
\textbf{\Large p. GERMANO BUSO}\\
	\textit{parroco dal 1985 al 1988}
\end{center}
\bigskip
\begin{center}
\textbf{\Large p. LORENZO GOTTARDELLO}\\
	\textit{parroco dal 1988 al 1997}
\end{center}
\bigbreak
\textit{Qualche traccia del mio vissuto a Trieste}
\medbreak
Provo mettere per iscritto qualcosa di quanto rimane in me, dopo molti anni, della mia
esperienza vissuta nella comunità parrocchiale di San Francesco in via Giulia. 
Non intendo stendere una cronaca, ma far emergere dalla mia memoria tracce che ritrovo ancora 
tanto vive di quel tempo trascorso a Trieste nella sezione di tempo: 1988-1997.
Avevo 52 anni quando approdai a Trieste. Venivo dall’esperienza pastorale romana protrattasi dal 
1973 al 1988 nella parrocchia di S. Marco, prima, e poi nella nuova parrocchia di S. Giuseppe da 
Copertino.
Non conoscevo per niente Trieste, né la realtà pastorale di S. Francesco. Conoscevo però i 
frati presenti allora: p. Tarcisio (per tutti: Ciccio); p. Luigi con cui avevo condiviso un anno a 
Roma; p. Giovanni, mio compagno di seminario; non conoscevo, se non di nome, il più giovane 
della comunità: p. Giambattista Bontempi.
Mi sentivo mandato dai superiori ad essere “guardiano” dei frati per essere insieme un segno di 
presenza francescana secondo l'ideale di San Francesco. In secondo luogo mi sentivo parroco 
inviato dal Vescovo per essere animatore e coordinatore delle attività pastorali nella parrocchia di 
San Francesco insieme con gli altri miei confratelli.
Nell'arco del mio mandato c'è stato un inserimento nuovo e alcuni avvicendamenti nella 
comunità dei frati. Nell'ottobre '89 è arrivato fra Armando con il compito di sacrestano e di prima 
accoglienza delle persone in sacrestia. Nel settembre '91 p. Giovanni, dopo 12 anni di permanenza è 
destinato a Sabaudia. Arriva in sostituzione numerica p. Adolfo della Torre: una meteora che 
compare solo per un anno. Così nel '92 arriva p. Ruggero Lotto desideroso di reinserirsi nella 
pastorale parrocchiale, mentre  è del novembre '93 la irremovibile scelta di p. Bontempi di andare a 
svolgere il suo ministero pastorale alla dipendenza di un vescovo diocesano. La sua partenza ha 
creato un vuoto pastorale in oratorio e tante difficoltà nel portare avanti l'animazione in quel settore. 
Nel settembre '94 il p. Provinciale ci ha inviato il giovane diacono p. Enzo Piovesan per prendere in 
mano la realtà dell'oratorio e l'animazione della pastorale giovanile. 
E' chiaro che ogni partenza lascia un po' di vuoto e ogni arrivo va accolto con le caratteristiche e 
propensioni pastorali che ognuno porta con sé, e questo richiede attenzione da parte di tutti per 
creare un nuovo spirito e un nuovo gioco di squadra.
\bigbreak
\underline{La Missione al popolo}
\medbreak
Quando sono arrivato a Trieste la diocesi era tutta presa dall'impegno della preparazione, 
ormai prossima, della Missione al popolo, che si è svolta poi in tutte le parrocchie dal 12 al 26 
febbraio '89.
Circa 400 missionari hanno ricevuto dal vescovo il crocefisso e sono stati inviati nelle varie 
parrocchie e realtà pastorali della diocesi per ravvivare la fede del popolo cristiano. Nella nostra 
parrocchia sono stati accolti e ospitati sette missionari: tre nostri confratelli e quattro suore 
francescane. Il coordinatore era il p. Francesco Faldani, già espulso dalla Cina e fondatore della 
missione in Corea. Nella prima settimana hanno visitato le famiglie riportando giorno per giorno 
informazione, richieste di vario genere e disponibilità a partecipare a cammini formativi e a servizi 
di volontariato nella comunità. Nella seconda settimana i missionari si sono concentrati nelle 
celebrazioni in chiesa e nelle catechesi comunitarie per raggruppamenti diversi, secondo le età.
Sono stati 15 giorni di presenza particolare dello Spirito Santo che ha ravvivato la fede nel 
cuore di molti. Frutto della Missione al popolo è stata la nascita di diversi gruppi di ascolto della 
Parola presso le famiglie e poi l'incontro settimanale sui testi biblici per celebrare l'eucarestia 
domenicale in maniera più partecipata e fruttuosa.
\bigbreak
\underline{Cammino neocatecumenale}
\medbreak
Prosperava in parrocchia da diversi anni la presenza del cammino neocatecumenale capace 
di avvicinare i lontani e di proporre un percorso impegnato attorno alla Parola, nella celebrazione 
viva dei sacramenti e nella carità, per essere poi annunciatori di Cristo a tutti gli uomini. Era una 
esperienza particolare che non poteva però proporsi come l'unica realtà della comunità parrocchiale, 
né d'altra parte porsi come cammino parallelo accanto a quello ordinario della comunità. 
Proprio alla fine della Missione al popolo e all'inizio del cammino verso la Pasqua il pastore 
della diocesi, il vescovo mons. Lorenzo Bellomi, ha emanato “Le Norme per il cammino 
neocatecumenale” da osservarsi in diocesi. Queste sono diventate da subito mio riferimento 
pastorale nei confronti del cammino in parrocchia. Le norme non sono mai state ufficialmente 
revocate. Le forti pressioni delle alte gerarchie dei neocatecumenali hanno indotto il Vescovo 
tacitamente a congelare tali Norme...
Con i responsabili del cammino neocatecumenale a San Francesco ci sono state diverse tensioni. 
Quando è intervenuto il loro grado gerarchico di livello superiore ho chiesto che mettessero in 
iscritto i punti essenziali per continuare il cammino in San Francesco, ai quali avrei data risposta 
scritta. Sul tenore della mia risposta i responsabili del Cammino neocatecumenale hanno deciso di 
spostare le comunità altrove. Il Vescovo Lorenzo Bellomi non mi ha mai mosso alcun rimprovero 
sul mio atteggiamento fermo, tenuto nei confronti del cammino neocatecumenale; d'altra parte gli 
avevo detto, dopo la confusione creatasi, che avrei agito decisamente secondo la mia coscienza.
\bigbreak
\underline{Coloritura francescana della parrocchia}
\medbreak
Dopo la Missione al popolo, oltre che impegnarci nel mettere al centro la Parola di Dio nel 
cammino della comunità, abbiamo incominciato a riflettere sulla coloritura francescana che doveva 
avere la parrocchia di San Francesco: perché era sotto la protezione del santo patrono d'Italia,  
perché era affidata ad una comunità di frati francescani, perché al suo interno aveva già altre piccole 
presenze francescane come la GIFRA e l'OFS che potevano essere sviluppate. Così abbiamo 
improntato un cammino di sensibilizzazione francescana durato diversi mesi. Nel contesto di questo 
percorso francescano abbiamo partecipato come parrocchia al grande pellegrinaggio regionale in 
Assisi, il 4 ottobre, per l'offerta dell'olio per la lampada votiva; abbiamo iniziato un cammino di 
formazione specifica per entrare nell'OFS e infine il 28 ottobre '90 abbiamo ricordato il 25\textdegree\ della 
parrocchia, che ricorreva il 3 ottobre, quando eravamo in pellegrinaggio in Assisi. 
C'è stato un triduo di preparazione; in una serata il p. Provinciale, p. Agostino Gardin 
(attuale vescovo di Treviso), ha sviluppato il tema: “Caratteristiche di una parrocchia animata da 
frati francescani”. 
Domenica 28 ottobre con semplicità abbiamo ricordato i 25 anni di vita della parrocchia con la 
presenza del vescovo mons. Lorenzo Bellomi, dei due ex parroci viventi: p. Fortunato e p. Olindo e 
dei sacerdoti del decanato. Al termine della Messa solenne il Vescovo ha benedetta la rinnovata 
sede della Fraternitas e i nuovi impianti sportivi, completamente rinnovati grazie alla tenacia di p. 
Giambattista. 
Circa la coloritura francescana della comunità parrocchiale successivamente è stata eretta la 
Milizia dell’Immacolata (di p. Kolbe) grazie allo zelo mariano di p. Ruggero. 
\bigbreak
\underline{L'oratorio e l'impegno per i giovani}
\medbreak
Quando sono giunto in via Giulia, la struttura dell'oratorio era utilizzata prevalentemente per 
la catechesi e per gli incontri delle comunità neocatecumenali. P. Giambattista cercava spazi 
ricreativi. L'ho sempre sostenuto in questo legittimo obiettivo e così lui trovando vari appoggi ha 
provveduto a rinnovare gli infissi e l'arredo nelle varie sale e ha affrontati gli impegnativi lavori di 
sistemazione dei campi da gioco con i bagni esterni.
La sua inaspettata partenza nel novembre '93 ha creato un vuoto di presenza, con tante 
difficoltà nel portare avanti l'animazione della pastorale giovanile e la gestione dell'oratorio.
Nel settembre '94 è arrivato il novello diacono p. Enzo Piovesan, incaricato a inserirsi in quel 
settore: lo ha fatto con le sue caratteristiche e con i suoi convincimenti. Conosciamo le difficoltà, 
lavorando tra i giovani, di trovare il giusto dosaggio tra le esigenze di stare insieme divertendosi in 
molteplici maniere e la proposta religiosa esplicita per aiutarli a crescere nella fede e maturare una 
scelta gioiosa di Cristo Gesù.
\bigbreak
\underline{Conclusione}
\medbreak
Ho accennato ai momenti più significativi della vita della comunità parrocchiale, ma nella 
vita quotidiana ci sono stati tanti altri aspetti ben più importanti per me condensati nelle relazioni 
interpersonali con tante persone e con tante famiglie. In queste relazioni sincere e profonde, 
condividendo gioie e sofferenze con chi il Signore mi faceva incontrare sul cammino, ho 
sperimentato la gioia e la fecondità del mio ministero sacerdotale. 
\begin{flushright}
\textit{fra Lorenzo Gottardello}
\end{flushright}
