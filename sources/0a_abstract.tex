\thispagestyle{empty}
\vspace*{15ex}
\begin{flushright}
\textit{La Fede e la Ragione sono come le due ali \\
con le quali lo spirito umano s'innalza \\
verso la contemplazione della verit\`a.\\}
\vspace{4ex}
\scriptsize{San Giovanni Paolo II.}
\end{flushright}
\cleardoublepage

\chapter*{Abstract}
\label{chap:abstract}
\addcontentsline{toc}{chapter}{Abstract}
\glsresetall
Negli ultimi anni, la grande crescita tecnologica dei sistemi wireless ha permesso lo sviluppo di nuovi
servizi ed applicazioni, disponibili in qualunque luogo per mezzo dei dispositivi mobili.
Molti di questi servizi, possono operare al meglio solo conoscendo la posizione del dispositivo.
Per questo motivo, i servizi basati sul posizionamento (LBS) hanno riscontrato negli ultimi anni un 
grande interesse, richiedendo ai moderni sistemi wireless, l'implementazione di sistemi di posizionamento
integrati nel dispositivo. 
Uno dei maggiori campi di applicazione degli LBS \`e sicuramente il settore dei trasporti, che include 
navigazione di veicoli, gestione degli incidenti, gestione del traffico e dell'assistenza stradale.
Un'altra importante applicazione \`e quella della sicurezza. I servizi di emergenza richiedono infatti 
che durante una telefonata ai numeri preposti (911 negli Stati Uniti e 112 in Europa),
la posizione del chiamante sia nota con elevata precisione alla centrale.
I servizi basati sul posizionamento hanno inoltre destato interesse in molti altri campi, 
come ad esempio per gli annunci pubblicitari basati sulla localit\`a dell'utente, per il controllo 
dello spostamento di persone o animali e per la navigazione pedonale nelle aeree urbane o dentro gli edifici.
Un'altra classe di servizi che sta diventando sempre pi\`u popolare, \`e quella delle assicurazioni a consumo.
Questo tipo di assicurazione viene applicata ai veicoli, e permette una determinazione del rischio
associato all'assicurato sulla base della quantit\`a di chilometri percorsi
e sul tipo di guida adottata dall'assicurato, portando eventualmente ad una diminuzione del premio.
Va detto che, molti dei servizi sopra citati necessitano di una stima accurata della posizione, mentre 
altri possono funzionare con una stima pi\`u approssimata.
Fin'ora, i servizi basati sull'utilizzo dei satelliti o \gls{gnss} hanno garantito prestazioni eccellenti 
nei dispositivi mobili. L'utilizzo dei dispositivi in ambienti ostili, come 
nelle citt\`a o all'interno di edifici peggiora per\`o le prestazioni dei sistemi di posizionamento GNSS, a causa 
della perdita di visibilit\`a dei satelliti e del tipo di propagazione a cammini multipli caratteristica 
di questi ambienti. Per ovviare a questo problema, sono stati studiati sistemi di 
posizionamento ibridi, basati sulla ricezione di segnali terrestri ad elevata potenza e ad ampio raggio,
come i segnali delle reti cellulari e i segnali Wi-Fi. Questi sistemi sfruttano la conoscenza della posizione delle 
stazioni radio base cellulari o degli access point Wi-Fi per ricavare la posizione del ricevitore, e possono 
entrare in funzione in supporto dei sistemi GNSS quando questi ultimi non sono disponibili. 
Questi sistemi di posizionamento non offrono per\`o 
le stesse prestazioni dei sistemi GNSS, e commettono un errore di posizionamento paragonabile alla dimensione della cella 
(alcuni Km) o alla portata degli access point Wi-Fi (alcune centinaia di metri). 
Per ottenere prestazioni paragonabili a quelle dei sistemi satellitari, si deve ricorrere ai sistemi di 
posizionamento basati sulla stima dei tempi di arrivo (TOA). 
Il nuovo standard \gls{3gpp} \gls{lte} basato su  un sistema di modulazione multi portante, risulta 
adeguato per questo scopo. LTE \`e infatti il primo standard mobile ad essere stato progettato con 
capacit\`a di posizionamento, grazie  alla trasmissione di particolari segnali di riferimento chiamati 
PRS. Lo standard LTE prevede inoltre la trasmissioni di segnali di riferimento ideati per altri scopi,
che possono essere sfruttati in maniera opportunistica per la localizzazione del ricevitore, 
nel caso in cui i PRS non siano disponibili. 
Il problema principale dei sistemi di posizionamento basati sulla stima dei TOA, \`e che il tempo di arrivo
del percorso diretto dev'essere riconosciuto e separato dal tempo di arrivo del segnale attraverso 
riflessioni multiple. Negli ambienti urbani o all'interno degli edifici, le componenti di segnale che arrivano al 
ricevitore, sono molto vicine tra loro, ed \`e quindi necessario trovare un modo per aumentare
la risoluzione della stima dei tempi di arrivo.
Per far fronte a questo problema, la tesi si \`e concentrata sull'utilizzo dei \gls{sra}, ovvero una serie
di strumenti matematici che possono essere utilizzati per la stima dei tempi di arrivo di un canale affetto da cammini 
multipli. Questi algoritmi prendono il nome dalla loro capacit\`a di distinguere componenti molto vicini tra di loro
in tempo e frequenza.
In questa tesi, sono stati realizzati dei framework per la stima dei tempi di 
arrivo (TOA) dei segnali di riferimento LTE che utilizzano i SRA per migliorare l'accuratezza della
stima dei tempi di arrivo. Gli algoritmi sviluppati sono stati applicati su dati LTE reali acquisiti
in uno scenario di propagazione a cammini multipli dal team di \gls{hsr} a su dati LTE emulati che sono stati 
gentilmente forniti dal tem del Centro Aereospaziale Tedesco (DLR).
Per dimostrare la loro efficacia, le prestazioni delle tecniche basate sugli SRA, sono state confrontate
con quelle di una tecnica di stima dei tempi di arrivo convenzionale sui segnali emulati.
Lo scenario in cui sono stati acquisiti i dati LTE reali \`e stato inoltre simulato con un software
per ray-tracing, ed i risultati sono stati confrontati con quelli ottenuti dagli algoritmi sviluppati
in questa tesi.
Il presente lavoro di tesi si articola in una prima parte teorica, dove vengono analizzate le principali tecniche di
posizionamento, lo standard LTE e gli strumenti matematici SRA che saranno 
utilizzati per la stima dei tempi di arrivo a partire dalla ricezione dei segnali di riferimento LTE, e 
in una seconda parte che presenta i nuovi framework per la stima dei tempi di arrivo di segnali LTE sviluppati in 
questo lavoro. Vengono infine presentati i risultati dell'utilizzo di questi strumenti su segnali LTE reali 
ed emulati. 

Nello specifico. la tesi \`e divisa nei seguenti capitoli.
\begin{itemize}
\item[-] Il capitolo \ref{chap:positioning} descrive i sistemi di posizionamento pi\`u utilizzati dal punto di vista 
delle grandezze fisiche implicate.
\item[-] Il capitolo \ref{chap:lte_phisical_layer} tratta nel dettaglio il sistema 
di modulazione multi-portante \gls{ofdm} ed il livello fisico dello standard LTE. Sono altres\'i descritti i principali segnali 
di riferimento dello standard LTE.
\item[-] Il capitolo \ref{chap:super_resolution_algorithms} analizza gli strumenti matematici chiamati 
\acrlong{sra}, su cui si basa il sistema di stima dei tempi di arrivo sviluppato in questa tesi.
\item[-] Il capitolo \ref{chap:framework_toa_estimation} presenta i diversi framework per la stima dei tempi di arrivo realizzati in questa tesi.
\item[-] Il capitolo \ref{chap:toa_estimation_real_data} riporta i risultati ottenuti applicando i framework su dati LTE reali misurati dal team 
di \gls{hsr}.
\item[-] Il capitolo \ref{chap:toa_estimation_emulated} riporta i risultati ottenuti applicando i framework su dati LTE emulati realizzati
dal team del DLR.
\item[-] Il capitolo \ref{chap:ray_tracing_simulation} descrive i risultati di una simulazione elettromagnetica sullo scenario dei dati LTE reali,
in modo da verificare i risultati ottenuti per via sperimentale.
\item[-] Infine, il capitolo \ref{chap:conclusions} riassume i risultati ottenuti in questo lavoro e propone delle nuove idee per future attivit\`a di ricerca.
\end{itemize}
Il lavoro che costituisce questa tesi \`e stato realizzato nel gruppo di ricerca per le telecomunicazioni dell'Universit\`a di Trieste sotto la
supervisione del Prof. Fulvio Babich e del Dr. Marco Driusso, che ringrazio cordialmente.