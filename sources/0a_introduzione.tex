\thispagestyle{empty}
\vspace*{15ex}
\begin{flushright}
\textit{Altissimu, onnipotente, bon Signore,\\
tue so’ le laude, la gloria e \\
l’honore et onne benedictione.\\}
\vspace{2ex}
\scriptsize{San Francesco d'Assisi}
\end{flushright}
\vfill
\small
\copyright\, 
\begin{minipage}[t]{10cm}
2015\\
PARROCCHIA DI SAN FRANCESCO\\
via Giulia, 72 - 34126 Trieste\\
Frati minori conventuali
\bigbreak 
\noindent Stampato ad uso interno
\end{minipage}
\cleardoublepage

\chapter*{Presentazione}
\label{chap:abstract}
\addcontentsline{toc}{chapter}{Presentazione}
È davvero un onore, per me, presentarvi questo libretto in memoria e celebrazione dei 
50 anni della nostra bella parrocchia di s. Francesco. Nella riunione del Consiglio Parrocchiale 
abbiamo raccolto alcune idee per celebrare e ricordare questo avvenimento, che cade giusto nella 
festa del nostro Patrono. E perché non riportare anche per iscritto i momenti salienti della crescita 
(strutturale, pastorale e spirituale) di una comunità di cristiani? Così è nato questo libricino, con 
l’unico intento di raccontare il cammino della comunità di san Francesco, attraverso i suoi 
protagonisti ancora viventi: pastori e fedeli.
\begin{center}
\bfseries
3 OTTOBRE 1965 – 4 OTTOBRE 2015: 50 anni di storia.
\end{center}

\noindent Qualcuno aggiungerebbe: “Di acqua sotto i ponti ne è passata tanta!”. E lo direbbe con molta 
ragione. Davvero la comunità di san Francesco in Trieste deve ringraziare tanto il Signore, e 
continuare a lodarlo e invocarlo perché non vengano mai a mancare una sapiente guida (Vescovi e 
frati), una fede pura ed umile, una carità ardente, una gioiosa speranza ed un popolo fedele ed unito. 
%\bigbreak

Il libretto che avete per le mani è lungi dall’essere una descrizione esaustiva ed ordinata di 
quanto è avvenuto nel territorio e nel cuore della gente: dovrete aspettare qualche altro buon 
parroco più dotato, paziente e libero del sottoscritto.
Qui ci limitiamo a presentarvi: la “Lettera fondazionale” della parrocchia, con la convenzione tra la 
Diocesi di Trieste e la Provincia religiosa dei Frati Minori Conventuali; l’augurio del Vescovo 
Santin; la Planimetria, con l’elenco aggiornato delle vie e dei numeri civici appartenenti alla 
parrocchia (sostanzialmente sottratti a san Giovanni Decollato); il racconto dei parroci  ancora 
viventi, delle suore Dimesse e quello di due gruppi di parrocchiani, dividendo il periodo in due parti 
(1965-1990; 1990-2015).
 
\noindent A lode e gloria di Dio!
\begin{flushright}
	\textit{fra Tiberio Zilio\\Guardiano e Parroco}
\end{flushright}