\documentclass[12pt,openright,twoside]{article}
\usepackage[a5paper]{geometry}
\usepackage[utf8]{inputenc}
\usepackage[autostyle]{csquotes}

\addtolength{\textwidth}{7mm}%aumento larghezza testo
\addtolength{\evensidemargin}{-1mm}%sposto la colonna di testo delle pagine pari a sinistra
\addtolength{\oddsidemargin}{1mm}%sposto la colonna di testo delle pagine dispari a destra
\setlength{\parindent}{0cm}

\begin{document}
\pagenumbering{gobble}
Brescia 15 luglio 2015 – S. Bonaventura\\
\begin{center}
50 ANNIVERSARIO DELLA PARROCCHIA DI \\S. FRANCESCO DI TRIESTE\\
1965 – 2015 
\end{center}
\noindent Nel mio pro-memoria per il Cinquantesimo di ordinazione sacerdotale (2010) “Cinquant’anni di grazia. (spigolature)” riferendomi ai nove anni di attività pastorale nella parrocchia di s. Francesco di Trieste (1967 – 1976) mi ero espresso in questo modo: 
\\\\

\begin{@empty}
\itshape
\lq\lq Quel po' di esperienza pastorale fatta presso la parrocchia dei santi Pietro e Paolo all’Eur (Roma) ancora fresco di ordinazione, e quella acquisita nella direzione dei chierici, sono venute buone quando i superiori mi hanno assegnato al convento di s. Francesco d’Assisi di Trieste con il duplice compito di superiore dei frati e di parroco della parrocchia dal vescovo assegnata ai religiosi conventuali.

Quella triestina è stata una stagione pastoralmente splendida, indimenticabile. Le primizie, infatti, sono sempre deliziose e piacevoli. Dal punto di vista canonico sono stato il primo religioso della comunità a ricoprire il ruolo di parroco della parrocchia triestina di s. Francesco. Se l’esperienza è stata positiva, e non solo per me, lo devo anche alla benevolenza del vescovo, monsignor Antonio Santin, alla cordialità dei sacerdoti diocesani e dei religiosi miei confratelli e, soprattutto, all’amicizia e all’affetto dei parrocchiani, che rimangono come perle preziose incastonate nella collana dei miei giorni. 

Trieste, città di confine e mitteleuropea, è sempre stata sensibile all’approccio ecumenico con altre esperienze religiose vicine e lontane: un percorso nel quale anche la nostra comunità si è efficacemente inserita, con varie iniziative.

Su altro versante, è stato impegnativo ma gratificante, il lavoro di rifacimento dell’interno della chiesa, con la rimozione dell’intonaco da tutta la superficie per liberare la pietra faccia a vista, che ha dato all’edificio un volto nuovo e con il rinnovato impianto di illuminazione, quasi anticipo e preludio dei successivi lavori, pure molto impegnativi, di sistemazione di tutta l’area presbiteriale. 

Del periodo triestino mi piace segnalare alcuni avvenimenti particolari, ancora vivi nel ricordo, come il viaggio ecumenico della diocesi a Istanbul, guidato da monsignor Santin e don Eugenio Ravignani, culminato nell’incontro con il Patriarca ortodosso Athenagoras I e il mio primo pellegrinaggio in Terra Santa, nel 1973, offertomi dai parrocchiani.\rq\rq
\end{@empty}
\\\\
Vengo ora a ripetere e rinnovare gli stessi sentimenti anche in questa circostanza, aggiungendovi qualche nota. 

Il decreto canonico di erezione a parrocchia veniva a confermare quanto già da circa venti anni si svolgeva in quella zona dello Scoglietto – Guardiella con il ritorno dei Frati Minori Conventuali. 

Non è stato perciò difficile inserirmi in quel contesto così bene avviato e consolidato con tutte le attività proprie di una comunità cristiana già ben collocata nel territorio diocesano. Preziosa la collaborazione con le Suore Dimesse. Cura e premura particolari del parroco si sono subito rivolte alla conoscenza diretta e personale nella fatica e nella gioia della benedizione e incontri familiari… 

Siamo agli inizi della riforma liturgica e della ripresa del cammino ecumenico: fare della parrocchia una comunità alimentata dalla liturgia feriale e festiva, curando specialmente le celebrazioni comunitarie dei battesimi, anniversari di matrimonio, prime comunioni, cresime, prime messe dei sacerdoti novelli dell’Istituto Teologico di Padova, fioretto di Maggio all’altare della Madonna con uno stuolo di irrequieti chierichetti o nei vari cortili in mezzo alle abitazioni, il pellegrinaggio annuale al santuario di Monte Grisa… e altri incontri di spiritualità e condivisione…. Un capitolo particolare merita la cura e attenzione all’oratorio per i ragazzi e i giovani. 

I sentimenti che hanno riempito la mia stagione pastorale triestina sono stati di dialogo, rispetto, collaborazione, fraternità e sincera amicizia, largamente ricambiati. Di tutto questo ancora ringrazio il Signore e auguro alla Comunità religiosa e parrocchiale di proseguire con lena sulla strada della nuova evangelizzazione secondo le incalzanti indicazioni di Papa Francesco. 
\begin{flushright}
P. Olindo M. Baldassa
\end{flushright}
\end{document}