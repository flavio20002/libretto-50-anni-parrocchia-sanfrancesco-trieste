\documentclass[12pt,a5paper]{article}
\usepackage[margin=19mm,paperwidth=314mm,paperheight=220mm]{geometry} %314=148.5 + 148.5 + 7 + 10 di abbondanza %220=210 + 10 di abbondanza
\usepackage[italian]{babel}
\usepackage[utf8]{inputenc}
\usepackage[T1]{fontenc}
\usepackage[sc]{mathpazo} % Palatino
\usepackage{lmodern} % Latin modern
\renewcommand*\rmdefault{ppl} %Palatino
\linespread{1.05}
\usepackage{graphicx}
\usepackage{tikz}
\usetikzlibrary{calc}
\usetikzlibrary{positioning}
\usepackage{fix-cm}
\usepackage{lipsum}
\usepackage{lettrine}

\def\lunghezzapagina{153.6mm}
\def\larghezzalaterale{7mm}

\definecolor{mio-verde}{rgb}{0.03, 0.47, 0.19}
\definecolor{mio-azzurro}{HTML}{6ea8b6}
\parindent=0pt

\begin{document}
\pagestyle{empty}

\newsavebox\Blurbbox
\sbox\Blurbbox{\begin{minipage}{110mm}
\lettrine[lines=4]{È}{} davvero un onore, per me, presentarvi questo libretto in memoria e celebrazione dei 
50 anni della nostra bella parrocchia di s. Francesco. Nella riunione del Consiglio Parrocchiale 
abbiamo raccolto alcune idee per celebrare e ricordare questo avvenimento, che cade giusto nella 
festa del nostro Patrono. E perché non riportare anche per iscritto i momenti salienti della crescita 
(strutturale, pastorale e spirituale) di una comunità di cristiani? Così è nato questo libricino, con 
l’unico intento di raccontare il cammino della comunità di san Francesco, attraverso i suoi 
protagonisti ancora viventi: pastori e fedeli.
Il libretto che avete per le mani è lungi dall’essere una descrizione esaustiva ed ordinata di 
quanto è avvenuto nel territorio e nel cuore della gente: dovrete aspettare qualche altro buon 
parroco più dotato, paziente e libero del sottoscritto.
Qui ci limitiamo a presentarvi: la “Lettera fondazionale” della parrocchia, con la convenzione tra la 
Diocesi di Trieste e la Provincia religiosa dei Frati Minori Conventuali; l’augurio del Vescovo 
Santin; la Planimetria, con l’elenco aggiornato delle vie e dei numeri civici appartenenti alla 
parrocchia (sostanzialmente sottratti a san Giovanni Decollato); il racconto dei parroci  ancora 
viventi, delle suore Dimesse e quello di due gruppi di parrocchiani, dividendo il periodo in due parti 
(1965-1990; 1990-2015).
A lode e gloria di Dio!
\begin{flushright}
	\textit{fra Tiberio Zilio}
\end{flushright}
\end{minipage}}

\begin{tikzpicture}[remember picture, overlay]
\node[minimum width=\lunghezzapagina,inner sep=0pt,anchor=east, outer sep=0pt, fill opacity=1.0, align=center] at (current page.east) {\includegraphics[width=\lunghezzapagina,height=\paperheight]{chiesa}};
\node[text width=\lunghezzapagina,anchor=north east, minimum width=\lunghezzapagina, minimum height=50mm, inner sep=0pt, outer sep=0pt, fill=cyan!50, fill opacity=0.7, text opacity=1, align=center] at (current page.north east) {%
};
\node[text width=\lunghezzapagina,anchor=north east, minimum width=\lunghezzapagina,yshift=-5mm, minimum height=45mm, inner sep=0pt, outer sep=0pt, text opacity=1, align=center] at (current page.north east) {%
\Large{\sffamily PARROCCHIA di SAN FRANCESCO in TRIESTE\par }\bigskip{\fontsize{25}{25}\selectfont\bfseries 1965-2015: 50 ANNI di VITA\par }\bigskip LA COMUNITÀ RACCONTA\par 
};
\node [minimum width=\lunghezzapagina, anchor=south east,align=center,fill opacity=0.3] at (current page.south east) {\includegraphics[width=8cm]{tau.png}};
\node[text width = \larghezzalaterale,anchor=south, minimum width=\larghezzalaterale, minimum height=\paperheight, inner sep=0pt, outer sep=0pt, fill=mio-azzurro, fill opacity=1, text opacity=1, align=center] at (current page.south) {%
};
\node[anchor=west, minimum width=\lunghezzapagina, minimum height=\paperheight, inner sep=0pt, outer sep=0pt, fill=mio-azzurro, fill opacity=1, text opacity=1, align=center] at (current page.west) {%
};
\node[anchor=north west, minimum width=148.6mm,xshift=+5mm, yshift=-24mm, inner sep=0pt, outer sep=0pt,text opacity=1, align=center] at (current page.north west) {%
\usebox\Blurbbox 
};
\draw[white, very thick] ($(current page.north west)+(0mm,-5mm)$) -- ++(1,0mm); %o
\draw[white, very thick] ($(current page.north west)+(\paperwidth,-5mm)$) -- ++(-1,0mm); %o
\draw[white, very thick] ($(current page.south west)+(0mm,5mm)$) -- ++(1cm,0mm); %u
\draw[white, very thick] ($(current page.south west)+(\paperwidth,5mm)$) -- ++(-1cm,0mm); %u
\draw[white, very thick] ($(current page.north east)+(-5mm,0mm)$) -- ++(0,-1cm); %r
\draw[white, very thick] ($(current page.north east)+(-5mm,-\paperheight)$) -- ++(0,1cm); %r
\draw[white, very thick] ($(current page.north west)+(5mm,0mm)$) -- ++(0,-1cm); %l
\draw[white, very thick] ($(current page.north west)+(5mm,-\paperheight)$) -- ++(0,1cm); %l
\draw[white, very thick] ($(current page.north)+(-3.5mm,0mm)$) -- ++(0,-3mm); %m
\draw[white, very thick] ($(current page.north)+(+3.5mm,0mm)$) -- ++(0,-3mm); %m
\draw[white, very thick] ($(current page.south)+(-3.5mm,0mm)$) -- ++(0,+3mm); %m
\draw[white, very thick] ($(current page.south)+(+3.5mm,0mm)$) -- ++(0,+3mm); %m
\end{tikzpicture}

\end{document}
