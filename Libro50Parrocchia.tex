\documentclass[11pt,titlepage,openright,twoside]{report}
\usepackage[a5paper]{geometry}
\usepackage[notlot]{tocbibind} % for having bibliography, list of figures, etc.. in the table of contents (nottoc,notlof,notlof for not displaying toc, lof, lot)
\usepackage[italian]{babel}
\usepackage[utf8]{inputenc}
\usepackage[T1]{fontenc}
\usepackage{lmodern}
\usepackage{url} % per gestione url nella bibliografia
\usepackage{amssymb} % per r reali
\usepackage{amscd} % per diagrammi
\usepackage{caption}
\usepackage{textcomp}
\usepackage{subcaption}
\usepackage{graphicx} % per immagini
\usepackage[margin=10pt,font=small,labelfont=bf,labelsep=endash]{caption} % didascalie personalizzate
\usepackage{fancyhdr}%per intestazione personalizzata
\usepackage[bookmarks,hidelinks]{hyperref} % for pdf index and links
\usepackage{enumerate} % for small roman letters in enumerate
\usepackage{algorithm} % for algorithms
\usepackage{algpseudocode}
%\usepackage[acronyms,nopostdot,nogroupskip,toc,nonumberlist]{glossaries}
\usepackage{booktabs}
\usepackage{tabularx}
\usepackage{cite} % for citation grouping
\usepackage{pdfpages}
\usepackage[autostyle]{csquotes}
\usepackage{multirow}
\usepackage{tabularx}
\usepackage{ltablex}
\usepackage{bigstrut}
\usepackage{needspace}

%%========================================================
%%     modifica dimensioni e formato del titolo dei capitoli
%%
\usepackage{titlesec,color}
\newcommand{\hsp}{\hspace{10pt}}
\titleformat{\chapter}[hang]{\LARGE\bfseries}{\thechapter\hsp}{0pt}{\thispagestyle{fancy}\LARGE\bfseries}
\titlespacing*{\chapter}{0cm}{-\topskip+20pt}{20pt}[0pt]
%%
%%========================================================

%%========================================================
%%     carattere
%%
%\renewcommand{\rmdefault}{ptm} %Times
%\renewcommand*\rmdefault{ppl} %Palatino
%\renewcommand*\rmdefault{cmr} %Computer Modern Roman (default)
%%
%%========================================================


\setlength{\parindent}{0cm} %Disabilita indentazione dei paragrafi

\setlength{\headheight}{15pt}
\renewcommand{\arraystretch}{1.4}
\renewcommand{\alglinenumber}[1]{\tiny#1:}

\def\tabularxcolumn#1{m{#1}} %center vertically
\newcolumntype{s}{>{\centering\arraybackslash\hsize=.5\hsize}X} %center horizontally

%%========================================================
%%     Permette di inserire un pdf come immagini
%%
\makeatletter
\newcounter{imagepage}
\newcommand*{\foreachpage}[2]{%
  \begingroup
    \sbox0{\includegraphics{#1}}%
    \xdef\foreachpage@num{\the\pdflastximagepages}%
  \endgroup
  \setcounter{imagepage}{0}%
  \@whilenum\value{imagepage}<\foreachpage@num\do{%
    \stepcounter{imagepage}%
    #2\relax
  }%
}
\makeatother
%%
%%========================================================

%%========================================================
%%     previene l'interruzione di pagina su tabelle lunghe in caso di utilizzo multirow
%%
\makeatletter
\def\@cline#1-#2\@nil{%
  \omit
  \@multicnt#1%
  \advance\@multispan\m@ne
  \ifnum\@multicnt=\@ne\@firstofone{&\omit}\fi
  \@multicnt#2%
  \advance\@multicnt-#1%
  \advance\@multispan\@ne
  \leaders\hrule\@height\arrayrulewidth\hfill
  \cr
  \noalign{\nobreak\vskip-\arrayrulewidth}}
\makeatother
%%
%%========================================================


%%========================================================
%%     modifica dimensioni layout larghezza pagina
%%
\addtolength{\textwidth}{7mm}%aumento larghezza testo
\addtolength{\hoffset}{-3.5mm}%diminuisco colonna per note a fianco per centrare testo allargato
\addtolength{\evensidemargin}{-1mm}%sposto la colonna di testo delle pagine pari a sinistra
\addtolength{\oddsidemargin}{1mm}%sposto la colonna di testo delle pagine dispari a destra
%%
%%========================================================

%%========================================================
%%            personalizzazione intestazione
%%
% i comandi seguenti impediscono la scrittura in maiuscolo
% dei nomi dei capitoli e dei paragrafi nelle intestazioni
\renewcommand{\chaptermark}[1]{\markboth{#1}{}}
\renewcommand{\sectionmark}[1]{\markright{\thesection\ #1}}
\fancyhf{} % rimuove l’attuale contenuto dell’intestazione
            % e del pi\‘e di pagina
%opzioni da mettere per twoside
\fancyhead[LE,RO]{\thepage}
\fancyhead[LO]{\nouppercase\rightmark}%\nouppercase elimina tutte le intestazioni maiuscole
\fancyhead[RE]{\nouppercase\leftmark}%\nouppercase elimina tutte le intestazioni maiuscole
%
%opzioni da mettere per oneside
%\fancyhead[RO]{\thepage}
%\fancyhead[LO]{\rightmark}
%
\renewcommand{\headrulewidth}{0.5pt}
\renewcommand{\footrulewidth}{0pt}
\addtolength{\headheight}{1.7pt} % riserva spazio per la linea
\fancypagestyle{plain}{%ridefinisce lo stile plain usato nelle pagine di chapter
\fancyhead{} % ignora, nello stile plain, le intestazioni
   \renewcommand{\headrulewidth}{0pt} % e la linea
   \fancyfoot[C]{\thepage}%per avere numero pagina anche pagina iniziale chapter
}
%%
%%========================================================

%%========================================================
%%   per imporre sillabazione
%%
\hyphenation{STSK}%
%%========================================================

%%========================================================
%%   comandi personalizzati
%%
\newcommand{\pedic}[2]{#1_{\text{#2}}}
%%========================================================

%\linespread{1.2}

\begin{document}
\raggedbottom % Prevent vertical justification
%%========================================================
%%  italian abstract and acknowledgements
\pagenumbering{roman}
\thispagestyle{empty}
\vspace*{15ex}
\begin{flushright}
\textit{Altissimu, onnipotente, bon Signore,\\
tue so’ le laude, la gloria e \\
l’honore et onne benedictione.\\}
\vspace{4ex}
\scriptsize{San Francesco d'Assisi}
\end{flushright}
\cleardoublepage

\chapter*{Introduzione}
\label{chap:abstract}
\addcontentsline{toc}{chapter}{Introduzione}
Una breve introduzione
\chapter*{Ringraziamenti}
\label{chap:ack}
\addcontentsline{toc}{chapter}{Ringraziamenti}

\vspace{4ex}
Qualche ringraziamento.
%%========================================================

%%========================================================
%%  indexes
\tableofcontents
%\listoffigures
%\printnoidxglossaries
%\listoftables
\cleardoublepage
%%========================================================

%%========================================================
%% thesis chapters
\pagenumbering{arabic} % normal page numbering
\pagestyle{fancy} % style with headnotes, rule, etc.
\addtolength{\headwidth}{12mm} % long rule
\chapter*{Saluto dell’Arcivescovo di Trieste}
\addcontentsline{toc}{chapter}{Saluto dell’Arcivescovo di Trieste}
Il 3 ottobre 1965, il mio amato predecessore S. E. Mons. Antonio Santin, con un decreto  
fondazionale, provvedeva ad erigere, nella Diocesi di Trieste, la parrocchia dedicata a San 
Francesco, affidandola alla cura pastorale dei Frati Conventuali. Il ricordo di quella data è motivo di 
gioiosa riconoscenza al Signore per tutte le grazie che, fino ad oggi, ha riservato alla comunità 
parrocchiale e a tutta la Chiesa di Trieste. La grazia del servizio generoso di tanti padri francescani 
che hanno speso le energie migliori del loro ministero per annunciare il Vangelo e per testimoniare 
la carità verso i poveri; la grazia di tantissimi bambini e bambine, di giovani che in parrocchia 
hanno imparato ad amare Dio e ad essere cittadini esemplari; la grazia di tante famiglie che nella 
comunità hanno rafforzato il loro amore e affrontato tante difficoltà; la grazia di tanti poveri, di 
tante persone sole e anziane, che nei fratelli e nelle sorelle della comunità di  san Francesco  hanno 
trovato consolazione e sostegno; la grazia di tanti collaboratori pastorali – catechisti, volontari, 
operatori della caritas… - che, nell’arco di questi cinquant’anni, hanno impreziosito la comunità 
con la testimonianza genuina delle loro opere buone; la grazia della Parola - quella di Gesù e quella 
che è Gesù - che ha convertito e rigenerato innumerevoli anime; la grazia dei sacramenti, soprattutto 
quelli dell’Eucaristia, del Battesimo e della Confessione che sono stati il nutrimento indispensabile 
per dare vita e identità autentiche alla comunità cristiana; la grazia della preghiera e quella della 
vocazione alla santità che Dio rivolge a tutti.
Cinquant’anni di vita cristiana vissuta insieme sotto la protezione di San Francesco; 
cinquant’anni che hanno registrato il miracolo quotidiano dell’amore misericordioso di Dio. Ora, 
doverosamente, la memoria deve aprire i cuori e le menti al futuro affinché la comunità si 
predisponga a viverlo - nella fede, nella speranza e nella carità cristiane - con la stessa fedeltà e il 
medesimo entusiasmo.
Nell’affidare la parrocchia di San Francesco alla materna protezione di Maria, Madre della 
Chiesa, colgo l’occasione per benedire tutti e di cuore.
\begin{flushright}
+Giampaolo Crepaldi\par
Arcivescovo - Vescovo di Trieste
\end{flushright}

\chapter*{Augurio del Ministro Provinciale\\ OFM Conv.}
\addcontentsline{toc}{chapter}{Augurio del Ministro Provinciale OFM Conv.}
\begin{center}
\bfseries
	Per i 50 anni della nostra Parrocchia\par San Francesco in Trieste
\end{center}
``Ti adoriamo, Signore nostro Gesù Cristo, qui e anche in tutte le chiese che sono nel mondo 
intero e ti benediciamo, perché con la tua santa croce hai redento il mondo” (Fonti francescane 
111). Mi affiora alla mente questa preghiera del Serafico Padre nella lieta ricorrenza dei 50 anni 
della Parrocchia di Trieste intitolata proprio a lui ed affidata dalla Diocesi all’animazione della 
nostra Provincia religiosa di Frati Minori Conventuali. È una preghiera che Francesco trovò ed 
elaborò sino a farla sua per poi consegnarla ai suoi frati e a quanti incontrava; un testo che gli 
faceva sobbalzare il cuore. Se ne intendeva di chiese il giovane Francesco: era nata tra queste, fra le 
più derelitte, la sua vocazione, giù nella piana di Assisi, ai margini della città, tra chiese 
abbandonate e corpi trascurati di lebbrosi. Entrambe queste realtà lo videro “restauratore” con 
mattoni e con lo sguardo, la vicinanza fattiva di chi dà dignità a chi l’aveva smarrita. Ognuna di 
queste “due” chiese, quella cadente di San Damiano e quella sofferente dei fratelli lebbrosi, aveva 
bisogno l’una dell’altra e rimandava al proprio tesoro: il Signore Gesù presente nella chiesa di 
mattoni e nei fratelli poveri, messi ai margini, scartati dai più. Iniziò così l’avventura che potremmo 
definire ``parrocchiale'' di Francesco che sino a quel momento della sua vita, si era concentrato 
unicamente su di sé, sui suoi sogni di autorealizzazione. Parafrasando un po' un testo noto: ``Dolce è 
sentire - è la scoperta di Francesco! - che non sono più solo, ma che son parte di una immensa vita, 
di una comunità di fratelli e sorelle, dono del Signore, del suo infinito amore…''
\bigbreak
``Francesco, va’, ripara la mia casa che, come vedi, è tutta in rovina'' (Fonti Francescane 
593). Cominciò così, con l’adesione semplice ed immediata all’invito del Crocifisso di San 
Damiano che gli aveva parlato trafiggendo di dolcezza il suo cuore, la vocazione di Francesco. 
Anche come manovale e muratore. Da allora, sino alla piena “conformitas”- conformità con il 
Crocifisso nel segno delle stigmate sul monte della Verna, quasi al finire dei suoi giorni, l’amore 
per il suo Signore fattosi uomo per noi, fu il centro del cuore di frate Francesco, dei suoi affetti, 
della sua predicazione. Meditava sempre, dice il suo primo biografo, fra Tommaso da Celano, 
“l’umiltà dell’incarnazione e la carità della passione” di Gesù, sino a piangere di commozione e di 
sofferenza quando constatava che “l’Amore non è amato”. Volle assomigliargli in tutto: con l’abito 
di sacco fatto a forma di croce, scegliendo il segno del Tau che riproduce la croce e, soprattutto, 
volendo per sé e i suoi frati una via di povertà e condivisione da vivere in letizia.

Fu solo più avanti che frate Francesco comprese che l’invito a “riparare la chiesa”
riguardava non solo gli edifici cadenti, ma l’edificio vivente della Chiesa. A dirla tutta, quella  del 
suo tempo era una Chiesa non molto edificante, anzi, bisognosa  di riforma, di un ritorno autentico 
al Vangelo, alla sua freschezza. Le chiese “poverelle” ed abbandonate incontrate da Francesco fuori 
Assisi erano come lo specchio della decadenza della Chiesa che attendeva  una vera e propria 
rifondazione. Frate Francesco e i suoi frati - i frati minori: un nome, un programma  - vi riuscirono: 
fu un “restauro” perfetto perché avvenne non “contro” la Chiesa, ma in obbedienza ad essa, 
accompagnandola a farle intravedere la bellezza e possibilità di vivere il Vangelo della fraternità e 
della minorità, della povertà e della letizia.

Eppure frate Francesco non si dimenticò i primi restauri che lo videro manovale e tuttofare 
nelle chiese povere e malandate: quando ne trovava in disordine, raccontano le Fonti Francescane, 
lui stesso cercava una scopa per spazzarle, si occupava del decoro delle tovaglie, delle suppellettili 
d’altare, coinvolgendo in questo progetto Chiara e le sue Povere Dame (in seguito denominate 
Clarisse). E così voleva facessero i frati. Perché? A motivo che qui - proprio in questa chiesa, qui in 
questa comunità ove ora mi trovo! - e in tutte le chiese che sono disseminate nel mondo, il Signore 
Gesù va adorato perché con la sua santa croce ha amato me e ogni persona di questa terra, ha 
salvato il mondo, gli uomini e le donne di tutti i tempi, offrendo tutto se stesso. Perché qui, in 
questa chiesa, qui in questa comunità parrocchiale, possiamo fare esperienza di lui, perché egli qui 
mi e ci raggiunge con il suo amore. Qui il Signore ci ha voluti per crescere insieme come famiglia 
parrocchiale.
\bigbreak
Nel celebrare i 50 anni della nostra Parrocchia di San Francesco in Trieste, auguro ai
confratelli e fedeli di poter vivere appieno gli atteggiamenti del cuore del Poverello di Assisi che 
della Comunità è il Patrono. La gioia di entrare, come singoli e come Comunità riunita in 
assemblea, nella bella chiesa di Via Giulia, per incontrare-lasciarsi incontrare ed adorare qui il 
Signore Gesù, per lasciarsi invadere dal suo amore che ci raggiunge nei sacramenti e nella preghiera 
assieme elevata alla Trinità santa; la letizia di scoprirsi una Fraternità parrocchiale radunata nel suo 
nome e da lui continuamente riconciliata; la responsabilità di “riparare la Chiesa”, anzitutto volendo 
bene ai frati, e alla Chiesa che ci è madre, aiutandoli con il proprio contributo di idee e la propria 
testimonianza nella ricerca del vero bene della Comunità; l’uscita in città, negli ambienti di vita ed 
anche in periferia, - come ci esorta Papa Francesco - per “fare misericordia” portando il Vangelo 
della fraternità a chi più soffre, a chi si sente messo ai margini.
La Parrocchia tutta - frati, suore, fedeli laici - sarà allora una Fraternità francescana che, meglio che 
può, con passione ed intelligenza,  segue e serve il Signore Gesù in letizia e minorità.
\bigbreak
Infine, per un compleanno così significativo, 50 anni!, corrono d’obbligo tre movimenti.
\begin{itemize}
\item Ringraziare per la strada sin qui compiuta. Ci sono storie, volti, esperienze di frati, 
parrocchiani che il cuore grande della Comunità contiene e sono indispensabili per tener viva 
l’identità e l’appartenenza. Una memoria cui ritornare, come ad una buona sorgente, con 
gratitudine, senza però troppo indulgere nella nostalgia ed affidando al Signore quanto ancora 
chiede di essere purificato e riconciliato pienamente.
\item Vivere il presente con fiducia e passione. La memoria grata del passato spinge, in ascolto 
attento di ciò che oggi lo Spirito Santo dice alla Chiesa, ad attuare in modo sempre più convinto 
l’adesione al Vangelo, calandolo nelle sfide e potenzialità dell’oggi. Con fede e passione.
\item Guardare il futuro con speranza. Lo può fare una comunità che coltivando una memoria 
grata si fida del Signore, della sua azione cui “nulla è impossibile”, senza cadere nello 
scoraggiamento e/o nella tentazione dei numeri e dell’efficienza, confidando solo nelle proprie 
forze. Infatti ogni momento della vita personale e comunitaria può essere kairòs ovvero “tempo di 
grazia” e di trasformazione, nel quale il Signore, visitandoci e rincuorandoci, ci fa avanzare per 
sentieri inediti e provvidenziali di pace e di bene.
\bigbreak
Buon compleanno, allora, carissima Comunità parrocchiale di San Francesco! Nella grazia sempre 
sorprendente e sovrabbondante del Signore e con la fraterna intercessione del Patrono, San 
Francesco d’Assisi,  ad multos annos!
\end{itemize}
\begin{flushright}
fr. Giovanni Voltan\par
ministro provinciale
\end{flushright}

\chapter{Documenti}
\label{chap:Documenti}
In questa sezione, vengono riportati alcuni documenti che riguardano la costituzione della nostra Parrocchia:

\begin{itemize}
\item \textbf{Lettera fondazionale}: Convenzione tra Diocesi di Trieste e Provincia Patavina dei frati minori conventuali.
\item \textbf{Riconoscimento civile}: Lettera della prefettura che accoglie la richiesta dell Diocesi per riconoscere a livello civile la nuova Parrocchia.
\item \textbf{Saluto del Vescovo}: L'augurio del Vescovo Santin.
\end{itemize}

%\newpage\mbox{}\newpage
\addcontentsline{toc}{section}{Lettera fondazionale}
\foreachpage{testi/lettera_fondazionale.pdf}{%
  \newpage   
  \begingroup 
    \centering
    \includegraphics[
      page=\value{imagepage},
      width=\textwidth,  
      height=\textheight,
    ]{testi/lettera_fondazionale.pdf}%
    \newpage
  \endgroup
}

%\newpage
%\begin{center}
%\includegraphics[
	%width=\textwidth,  
	%height=\textheight,
	%]{immagini/i_due_vicari.jpg}%
%\end{center}
%\newpage

\newpage
\begin{center}
\includegraphics[
	width=1.0\textwidth
	]{testi/lettera_fondazionale10.pdf}%
\end{center}
\vspace*{10mm}
\textit{Il 3 maggio 1994 è stata rinnovata la Convenzione tra la Diocesi di Trieste e la Provincia Patavina OFM Conv (ora Provincia Italiana di S. Antonio di Padova) conforme le nuove indicazioni del Concilio Ecumenico Vaticano II.}

\foreachpage{testi/riconoscimento_civile.pdf}{%
  \newpage   
  \begingroup 
    \centering
    \includegraphics[
      page=\value{imagepage},
      width=\textwidth,  
      height=\textheight,
    ]{testi/riconoscimento_civile.pdf}%
    \newpage
  \endgroup
}

\newpage
\begin{center}
\includegraphics[
	width=\textwidth,  
	height=\textheight
	]{immagini/saluto_dal_vescovo.jpg}%
\end{center}
\newpage
\begin{center}
\includegraphics[
	width=\textwidth,  
	height=\textheight,
	keepaspectratio
	]{immagini/FotoParrocchia.jpg}%
\end{center}
\newpage
\chapter{Stradario della Parrocchia}
\label{chap:Stradario}

\begin{center}
\textbf{PARROCCHIA DI SAN FRANCESCO D’ASSISI}\\[0.5cm]
\textbf{STRADARIO 2011}
\end{center}
\begin{center}
  \scriptsize
	\begin{tabularx}{\textwidth}{| s | s | X |}
		\hline
		\multirow{2}{*}[-0pt]{\textbf{Nome via}} & 
		\textbf{inizio/fine dispari}&\textbf{numeri dispari}\\
		\cline{2-3}
		&\textbf{inizio/fine pari}&\textbf{numeri pari}\\
		\hline
		\multirow{2}{*}[-5pt]{via Berchet} &
		\multirow{2}{*}[-5pt]{tutta} &
		1, 3, 5, 7, 9, 11, 13, 15, 17, 19, 21, 23, 25, 27, 29, 31, 33, 35, 37\\
		\cline{3-3}
		&&2, 4, 6, 8, 10, 12, 12/1, 12/2, 14, 16, 18, 20, 22, 26\\
		\hline
		\multirow{2}{*}{via dei Bonomo} &
		\multirow{2}{*}{tutta} &
		1, 3, 5, 9, 11, 13, 15, 15/1, 17, 19\\
		\cline{3-3}
		&&2, 2/1, 2/2, 2/3, 2/4, 4\\
		\hline
		rotonda del Boschetto&
		tutta meno lato est &
		1, 2, 3, 3/1, 6\\
		\hline
		\multirow{2}{*}{androna Cesarotti} &
		\multirow{2}{*}{tutta} &
		1, 3, 5\\
		\cline{3-3}
		&&2, 4, 6, 8, 10\\
		\hline
		via di Cologna &
		lato dispari fra vie Kandler e Edera &
		25, 27, 27/1, 29, 29/1, 31, 33, 35, 37, 39, 41\\
		\hline
		\multirow{2}{*}{via dei Cunicoli} &
		\multirow{2}{*}{tutta} &
		1, 3, 5, 7, 9, 11, 13\\
		\cline{3-3}
		&&2, 4, 6, 8, 10\\
		\hline
		vicolo dell'Edera &
		dispari inizio via &
		1\\
		\hline
		\multirow{2}{*}{scala Ferolli} &
		\multirow{2}{*}{tutta} &
		1, 3\\
		\cline{3-3}
		&&4\\
		\hline
		\multirow{2}{*}{via Ferrari} & 
		lato dispari tutti&1, 3, 5, 7, 9, 11\\
		\cline{2-3}
		&lato pari inizio via&2\\
		\hline
		\multirow{2}{0.5\linewidth}[-30pt]{\centering via Giulia (finisce in rotonda del Boschetto)} &
		\multirow{2}{0.5\linewidth}[-30pt]{\centering da piazza Volontari Giuliani a fine} &
		41, 43, 45, 47, 49, 51, 53, 55, 57, 59, 61, 63, 65, 67, 69, 71, 73, 75, 75/1, 75/2, 75/3, 77, 79, 81, 83, 85\\
		\cline{3-3}
		&&24, 26, 28, 30, 32, 34, 36, 38, 48, 50, 52, 54, 54/1, 56, 58, 58/1, 60, 62, 64, 66, 68, 70, 72, 74, 76, 78, 80, 82, 84, 86, 88, 90,
		92, 94, 94/1, 96, 96/1, 98, 100, 102, 104, 108\\
		\hline
		\multirow{2}{0.5\linewidth}{\centering via Kandler (finisce in via Giulia) } &
		\multirow{2}{0.5\linewidth}{\centering da via di Cologna a fine} &
		7, 9, 11, 13, 15\bigstrut\\
		\cline{3-3}
		&&8, 10, 12, 14, 16\bigstrut\\
		\hline
		\multirow{2}{*}{via Margherita} &
		\multirow{2}{*}{tutta} &
		1, 5, 7, 9, 11, 13, 15, 19, 21, 23, 25\\
		\cline{3-3}
		&&2, 4, 4/1, 4/2, 4/3, 6, 8, 10\\
		\hline
		\multirow{2}{*}{via Mercantini} &
		\multirow{2}{*}{tutta} &
		1, 3, 5, 7, 9, 11, 11/1, 13\\
		\cline{3-3}
		&&2, 2/1, 4, 6, 8, 10, 12, 14\\
		\hline
		\multirow{2}{*}{scala al Monticello} &
		\multirow{2}{*}{tutta} &
		1, 3\\
		\cline{3-3}
		&&2, 4\\
		\hline
		\multirow{2}{*}{via dell’Oliveto} &
		\multirow{2}{*}{tutta} &
		1, 3, 5, 7, 9, 11\\
		\cline{3-3}
		&&2, 4\\
		\hline
		\multirow{2}{*}{via del Pilone} &
		\multirow{2}{*}{tutta} &
		1, 3, 5\\
		\cline{3-3}
		&&2, 4\\
		\hline
		\multirow{2}{*}[-5pt]{via Pindemonte} &
		\multirow{2}{*}[-5pt]{tutta} &
		1, 3, 5, 5/1, 7, 7/1, 7/2, 7/3, 9, 9/1, 9/2, 9/3, 9/4, 9/5, 11, 13\\
		\cline{3-3}
		&&2, 2/3, 4, 6, 8, 8/1, 8/2, 10, 10/1, 10/2, 12, 14\\
		\hline
		\multirow{2}{*}{via Pisoni} &
		\multirow{2}{*}{tutta} &
		1, 3, 5, 7, 9, 11, 13, 17\\
		\cline{3-3}
		&&2, 4, 6, 10, 10/1, 12, 14, 18\\
		\hline
		vicolo dei Roveri&
		lato pari &
		2, 4, 6, 8, 10, 14, 16\\
		\hline
		\multirow{2}{*}[-5pt]{androna S. Cilino} &
		\multirow{2}{*}[-5pt]{tutta} &
		1, 3, 5, 7, 9, 11, 13, 15, 17, 19, 21, 23, 25\\
		\cline{3-3}
		&&2, 8, 10, 12, 14, 16, 18, 20, 22, 24, 26, 28, 30\\
		\hline
		\multirow{2}{*}{via S. Cilino} & 
		da inizio a v. S. Primo&1, 3, 5, 7, 9, 11, 13, 21, 23, 25, 27, 29\\
		\cline{2-3}
		&da inizio a v.Roveri&2, 4, 6\\
		\hline
		\multirow{2}{*}{via S. Donato} &
		\multirow{2}{*}{tutta} &
		1, 3, 5, 7, 9, 11, 13, 15, 21, 23\\
		\cline{3-3}
		&&2, 4, 6, 8, 10, 12, 14, 16, 20, 24, 26, 28\\
		\hline
		\multirow{2}{*}{via S. Felice} &
		\multirow{2}{*}{tutta} &
		1, 3, 5, 7\\
		\cline{3-3}
		&&2, 4, 6, 8, 10, 12, 14, 16\\
		\hline
		campo S. Luigi&
		lato v. Pindemonte &
		1, 1/1, 2, 3, 3/1, 4, 5, 6, 7\\
		\hline
		scala S. Luigi&
		lato pari&
		2\\
		\hline
		via S. Primo&
		dispari inizio via&
		1\\
		\hline
		\multirow{2}{0.5\linewidth}[-5pt]{\centering pendice dello Scoglietto} & 
		lato dispari tutti&1, 3, 3/1, 3/2, 5, 5/1, 5/2, 5/3, 5/4, 5/5, 5/6, 7, 7/1, 9, 11, 13, 13/1, 15, 17\\
		\cline{2-3}
		&lato pari da inizio a vicolo dell’Edera&2, 4, 6, 8, 10, 12, 14, 16, 18\\
		\hline
		\multirow{2}{0.5\linewidth}{\centering via dello Scoglio (inizia in via Giulia)} &
		\multirow{2}{0.5\linewidth}{\centering da inizio a pendice Scoglietto} &
		1, 3, 5, 7, 9, 11, 13, 15, 17, 19, 21, 23, 25, 27, 29, 31, 33, 35, 37, 39, 51, 59, 61, 63, 67, 69, 71, 73, 75, 77, 83, 85, 87, 89, 
		91, 95, 97, 99, 103, 1077\\
		\cline{3-3}
		&&2, 4, 6, 8, 12, 14, 14/1, 16, 18, 20\\
		\hline
		\multirow{2}{0.5\linewidth}[-5pt]{\centering viale XX Settembre (finisce in via Bonomo)} &
		\multirow{2}{0.5\linewidth}[-5pt]{\centering da piazza Volontari Giuliani a fine} &
		77, 79, 81, 83, 85, 87, 89, 89/1, 93, 97, 97/1, 101, 103\\
		\cline{3-3}
		&&72, 74, 76, 78, 80, 82, 84, 86, 88, 90, 92, 94, 96, 98, 100, 100/1, 102, 102/1, 104\\
		\hline
	\end{tabularx}
\end{center}
\newpage
\chapter{I Parroci raccontano}
\label{chap:Testimonianze}

\begin{center}
\textbf{\Large p. FORTUNATO ZORZINI}\\
	\textit{Amministratore parrocchiale dal 1965 al 1967}
\end{center}
%\textbf{Vicari: p. Stanislao Sgarbossa e p. Salvatore Stagni}
\bigskip
\begin{center}
\textbf{\Large p. OLINDO BALDASSA racconta...}\\
	\textit{parroco dal 1967 al 1976}
\end{center}
%\textbf{Vicari: p. Stanislao Sgarbossa e p. Salvatore Stagni}
\bigbreak
\begin{flushright}
Brescia, 3 ottobre 1976\\
Festa di S. Bonaventura
\end{flushright}
\noindent Nel mio pro-memoria per il Cinquantesimo di ordinazione sacerdotale (2010) “Cinquant’anni di grazia. (spigolature)” riferendomi ai nove anni di attività pastorale nella parrocchia di s. Francesco di Trieste (1967 – 1976) mi ero espresso in questo modo:
\bigbreak 
\enquote{\itshape Quel po' di esperienza pastorale fatta presso la parrocchia dei santi Pietro e Paolo all’Eur (Roma) ancora fresco di ordinazione, e quella acquisita nella direzione dei chierici, sono venute buone quando i superiori mi hanno assegnato al convento di s. Francesco d’Assisi di Trieste con il duplice compito di superiore dei frati e di parroco della parrocchia dal vescovo assegnata ai religiosi conventuali.

Quella triestina è stata una stagione pastoralmente splendida, indimenticabile. Le primizie, infatti, sono sempre deliziose e piacevoli. Dal punto di vista canonico sono stato il primo religioso della comunità a ricoprire il ruolo di parroco della parrocchia triestina di s. Francesco. Se l’esperienza è stata positiva, e non solo per me, lo devo anche alla benevolenza del vescovo, monsignor Antonio Santin, alla cordialità dei sacerdoti diocesani e dei religiosi miei confratelli e, soprattutto, all’amicizia e all’affetto dei parrocchiani, che rimangono come perle preziose incastonate nella collana dei miei giorni. 

Trieste, città di confine e mitteleuropea, è sempre stata sensibile all’approccio ecumenico con altre esperienze religiose vicine e lontane: un percorso nel quale anche la nostra comunità si è efficacemente inserita, con varie iniziative.

Su altro versante, è stato impegnativo ma gratificante, il lavoro di rifacimento dell’interno della chiesa, con la rimozione dell’intonaco da tutta la superficie per liberare la pietra faccia a vista, che ha dato all’edificio un volto nuovo e con il rinnovato impianto di illuminazione, quasi anticipo e preludio dei successivi lavori, pure molto impegnativi, di sistemazione di tutta l’area presbiteriale. 

Del periodo triestino mi piace segnalare alcuni avvenimenti particolari, ancora vivi nel ricordo, come il viaggio ecumenico della diocesi a Istanbul, guidato da monsignor Santin e don Eugenio Ravignani, culminato nell’incontro con il Patriarca ortodosso Athenagoras I e il mio primo pellegrinaggio in Terra Santa, nel 1973, offertomi dai parrocchiani.}
\bigbreak

Vengo ora a ripetere e rinnovare gli stessi sentimenti anche in questa circostanza, aggiungendovi qualche nota. 

Il decreto canonico di erezione a parrocchia veniva a confermare quanto già da circa venti anni si svolgeva in quella zona dello Scoglietto – Guardiella con il ritorno dei Frati Minori Conventuali. 

Non è stato perciò difficile inserirmi in quel contesto così bene avviato e consolidato con tutte le attività proprie di una comunità cristiana già ben collocata nel territorio diocesano. Preziosa la collaborazione con le Suore Dimesse. Cura e premura particolari del parroco si sono subito rivolte alla conoscenza diretta e personale nella fatica e nella gioia della benedizione e incontri familiari… 

Siamo agli inizi della riforma liturgica e della ripresa del cammino ecumenico: fare della parrocchia una comunità alimentata dalla liturgia feriale e festiva, curando specialmente le celebrazioni comunitarie dei battesimi, anniversari di matrimonio, prime comunioni, cresime, prime messe dei sacerdoti novelli dell’Istituto Teologico di Padova, fioretto di Maggio all’altare della Madonna con uno stuolo di irrequieti chierichetti o nei vari cortili in mezzo alle abitazioni, il pellegrinaggio annuale al santuario di Monte Grisa… e altri incontri di spiritualità e condivisione…. Un capitolo particolare merita la cura e attenzione all’oratorio per i ragazzi e i giovani. 

I sentimenti che hanno riempito la mia stagione pastorale triestina sono stati di dialogo, rispetto, collaborazione, fraternità e sincera amicizia, largamente ricambiati. Di tutto questo ancora ringrazio il Signore e auguro alla Comunità religiosa e parrocchiale di proseguire con lena sulla strada della nuova evangelizzazione secondo le incalzanti indicazioni di Papa Francesco. 
\begin{flushright}
\textit{P. Olindo M. Baldassa}
\end{flushright}
\bigbreak
\textit{Riteniamo utile, a completamento di quanto ci ha raccontato p. Olindo, riportare il suo saluto alla parrocchia, nel giorno della sua partenza (3 ottobre 1976):}
\bigbreak
\begin{flushright}
\needspace{3\baselineskip}
Trieste, 3 ottobre 1976
\end{flushright}
Miei Fratelli,\par
la domenica 15 Ottobre 1967 mi presentavo a voi assumendo il servizio pastorale di questa comunità parrocchiale, oggi celebro per voi e con voi la S. Messa passando ad altre mani il mio mandato.
Sono passati 9 anni: possono essere considerati molti o pochi a seconda del punto di vista dal quale ci si colloca, è un normale avvicendamento, sia pure non richiesto, sono stati comunque vissuti 
questi nove anni, mi sembra di poter dire con intensità di operosità, di conoscenza reciproca, di stima, di crescita nella comune vocazione alla risposta della parola di Dio.
Mi sia consentito, sia pure brevemente, di ricordare qualche tratto del cammino svolto insieme in questi anni:
in continuità di opere con il mio predecessore mi sono anzitutto dedicato a rendere sempre più accogliente la chiesa, come luogo di riunione e convegno di preghiera: 
ecco i banchi nuovi offerti in memoria dei vostri defunti, le suppellettili e arredi per una nobile e dignitosa celebrazione delle sacre funzioni, il riscaldamento della chiesa esteso anche al 
Franciscanum e Oratorio, il rimaneggiamento della pavimentazione, la ristrutturazione dell’impianto di illuminazione e i sei grandi pendenti, l’impianto di amplificazione e finalmente
il grande, lungo, faticoso restauro della superficie muraria interna con la demolizione degli intonaci e la pulitura delle superfici dei conci di pietra arenaria e successiva fugatura così
da portare a vista tutta la superficie interna e dare un volto nuovo, vibrante e austero, a tutta l'architettura e significare così, almeno per immagine, la vera struttura della chiesa, corpo
mistico di Cristo, costituita da pietre vive.
La non indifferente somma di denaro per far fronte e sostenere le spese per detti lavori è stata, reperibile grazie alla alla vostra generosità e ad una accorta amministrazione senza lasciare
attualmente alcuna passività.
Ma la mia cura maggiore si è rivolta non tanto alle opere da fare, sia pure utili o necessarie, quanto alle persone come componenti vive della comunità cristiana dirigendo in particolare 
la mia azione verso il ministero di\\
a) dispensatore della parola intesa come evangelizzazione, infatti quante prediche avete sentito... e le istruzioni nelle diverse Associazioni (Azione Cattolica, Terz'Ordine, Milizia, 
S. Vincenzo, giornate particolari con tridui, novene ecc.... assemblee, incontri di mamme e papà ... e la preparazione catechistica ai bambini per i sacramenti della iniziazione cristiana e 
agli adulti per vivificare il senso di consapevolezza e responsabilità dei loro doveri cristiani nei confronti dei figli....da questa maturazione ecco l'aspetto di\\
b) dispensatore dei sacramenti: come veri incontri con la grazia di Dio che fa rinascere nel battesimo, corrobora nella Cresima, nutre nell'Eucarestia, perdona nella riconciliazione, 
consacra l'amore nel matrimonio, risana e apre a speranza nell'unzione degli infermi.\\Lo sforzo ancora e l'impegno all'interno di questa famiglia di farmi\\
c) dispensatore di carità, nell'accoglienza e nel servizio tutte le persone nelle visite alle famiglie; gli ammalati e i vecchi in modo particolare, nella preghiera fatta insieme qui in chiesa,
nelle vostre abitazioni e anche nei vostri cortili come non ricordare il maggio dell'anno scorso, l'anno santo, e l'attenzione ecumenica fatta di conoscenza 
e rispettosa deferenza verso i Fratelli separati...\\
Intenzione esplicita e come forza motrice in questi anni mi è stata quella di creare tra noi tutti un clima di famiglia, rendere la chiesa veramente la casa di tutti, la casa comune, dove ognuno si
sentisse a suo agio, si sentisse bene, e vi potesse aprire serenamente e fiduciosamente il suo animo e ...
quindi le Associazioni Cattoliche sì, ma volutamente non ho prestato attenzione a gruppi o gruppuscoli che avessero potuto essere elementi di separazione e di isolamento nel tessuto della
comunità. Con particolare predilezione ho curato i chierichetti, sempre bravi e numerosi alle celebrazioni; il settore giovani è stato quello che per diverse cause, 
sia logistiche di spazi e ambienti che per mancanze di personale ha più sofferto.
Ai momento quindi di accomiatarmi, ringrazio tutti voi per le preziose e attente collaborazione prestate, a sviluppare questi miei intendimenti che il Signore Dio giudicherà nella 
loro realizzazione.
Essendo la parrocchia una parte della chiesa locale ossia della Diocesi rivolgo il mio deferente pensiero al Vescovo che ora la governa e in particolare a Mons. Santin con il quale
ho maggiormente intrattenuto relazioni di collaborazione e di rispettosa e filiale vicinanza.
È per me doveroso ricordare e accomunare nel ringraziamento tutti i confratelli che hanno con me condiviso in questi anni (anche se ora si trovano lontano da noi) 
la responsabilità e le fatiche dell'apostolato nei diversi settori pastorali e qui rappresentati nella concelebrazione da P. Luigi... 
Il mio successore: non spetta a me presentarlo, lo farà ufficialmente il Vescovo domani durante la Messa Vespertina; mi sia consentito solamente di dire che egli viene in mezzo a voi ricco di
esperienza pastorale avendo già retto per undici anni la parrocchia del Tempio Votivo a Verona, e secondariamente, un fatto personale: il P. Innocenzo era allora sacerdote novello a 
Camposampiero e per quei miei primi anni di Seminario è stato il mio Direttore Spirituale...una certa continuità...
\begin{flushright}
\includegraphics[scale=0.16]{immagini/firmaBaldassa.jpg}
\end{flushright}
%\newpage
\bigskip
\begin{center}
\textbf{\Large p. INNOCENZO BORDIN}\\
	\textit{parroco dal 1976 al 1985}
\end{center}
\bigskip
\begin{center}
\textbf{\Large p. GERMANO BUSO}\\
	\textit{parroco dal 1985 al 1988}
\end{center}
\bigskip
\begin{center}
\textbf{\Large p. LORENZO GOTTARDELLO}\\
	\textit{parroco dal 1988 al 1997}
\end{center}
\bigbreak
\textit{Qualche traccia del mio vissuto a Trieste}
\medbreak
Provo mettere per iscritto qualcosa di quanto rimane in me, dopo molti anni, della mia
esperienza vissuta nella comunità parrocchiale di San Francesco in via Giulia. 
Non intendo stendere una cronaca, ma far emergere dalla mia memoria tracce che ritrovo ancora 
tanto vive di quel tempo trascorso a Trieste nella sezione di tempo: 1988-1997.
Avevo 52 anni quando approdai a Trieste. Venivo dall’esperienza pastorale romana protrattasi dal 
1973 al 1988 nella parrocchia di S. Marco, prima, e poi nella nuova parrocchia di S. Giuseppe da 
Copertino.
Non conoscevo per niente Trieste, né la realtà pastorale di S. Francesco. Conoscevo però i 
frati presenti allora: p. Tarcisio (per tutti: Ciccio); p. Luigi con cui avevo condiviso un anno a 
Roma; p. Giovanni, mio compagno di seminario; non conoscevo, se non di nome, il più giovane 
della comunità: p. Giambattista Bontempi.
Mi sentivo mandato dai superiori ad essere “guardiano” dei frati per essere insieme un segno di 
presenza francescana secondo l'ideale di San Francesco. In secondo luogo mi sentivo parroco 
inviato dal Vescovo per essere animatore e coordinatore delle attività pastorali nella parrocchia di 
San Francesco insieme con gli altri miei confratelli.
Nell'arco del mio mandato c'è stato un inserimento nuovo e alcuni avvicendamenti nella 
comunità dei frati. Nell'ottobre '89 è arrivato fra Armando con il compito di sacrestano e di prima 
accoglienza delle persone in sacrestia. Nel settembre '91 p. Giovanni, dopo 12 anni di permanenza è 
destinato a Sabaudia. Arriva in sostituzione numerica p. Adolfo della Torre: una meteora che 
compare solo per un anno. Così nel '92 arriva p. Ruggero Lotto desideroso di reinserirsi nella 
pastorale parrocchiale, mentre  è del novembre '93 la irremovibile scelta di p. Bontempi di andare a 
svolgere il suo ministero pastorale alla dipendenza di un vescovo diocesano. La sua partenza ha 
creato un vuoto pastorale in oratorio e tante difficoltà nel portare avanti l'animazione in quel settore. 
Nel settembre '94 il p. Provinciale ci ha inviato il giovane diacono p. Enzo Piovesan per prendere in 
mano la realtà dell'oratorio e l'animazione della pastorale giovanile. 
E' chiaro che ogni partenza lascia un po' di vuoto e ogni arrivo va accolto con le caratteristiche e 
propensioni pastorali che ognuno porta con sé, e questo richiede attenzione da parte di tutti per 
creare un nuovo spirito e un nuovo gioco di squadra.
\bigbreak
\underline{La Missione al popolo}
\medbreak
Quando sono arrivato a Trieste la diocesi era tutta presa dall'impegno della preparazione, 
ormai prossima, della Missione al popolo, che si è svolta poi in tutte le parrocchie dal 12 al 26 
febbraio '89.
Circa 400 missionari hanno ricevuto dal vescovo il crocefisso e sono stati inviati nelle varie 
parrocchie e realtà pastorali della diocesi per ravvivare la fede del popolo cristiano. Nella nostra 
parrocchia sono stati accolti e ospitati sette missionari: tre nostri confratelli e quattro suore 
francescane. Il coordinatore era il p. Francesco Faldani, già espulso dalla Cina e fondatore della 
missione in Corea. Nella prima settimana hanno visitato le famiglie riportando giorno per giorno 
informazione, richieste di vario genere e disponibilità a partecipare a cammini formativi e a servizi 
di volontariato nella comunità. Nella seconda settimana i missionari si sono concentrati nelle 
celebrazioni in chiesa e nelle catechesi comunitarie per raggruppamenti diversi, secondo le età.
Sono stati 15 giorni di presenza particolare dello Spirito Santo che ha ravvivato la fede nel 
cuore di molti. Frutto della Missione al popolo è stata la nascita di diversi gruppi di ascolto della 
Parola presso le famiglie e poi l'incontro settimanale sui testi biblici per celebrare l'eucarestia 
domenicale in maniera più partecipata e fruttuosa.
\bigbreak
\underline{Cammino neocatecumenale}
\medbreak
Prosperava in parrocchia da diversi anni la presenza del cammino neocatecumenale capace 
di avvicinare i lontani e di proporre un percorso impegnato attorno alla Parola, nella celebrazione 
viva dei sacramenti e nella carità, per essere poi annunciatori di Cristo a tutti gli uomini. Era una 
esperienza particolare che non poteva però proporsi come l'unica realtà della comunità parrocchiale, 
né d'altra parte porsi come cammino parallelo accanto a quello ordinario della comunità. 
Proprio alla fine della Missione al popolo e all'inizio del cammino verso la Pasqua il pastore 
della diocesi, il vescovo mons. Lorenzo Bellomi, ha emanato “Le Norme per il cammino 
neocatecumenale” da osservarsi in diocesi. Queste sono diventate da subito mio riferimento 
pastorale nei confronti del cammino in parrocchia. Le norme non sono mai state ufficialmente 
revocate. Le forti pressioni delle alte gerarchie dei neocatecumenali hanno indotto il Vescovo 
tacitamente a congelare tali Norme...
Con i responsabili del cammino neocatecumenale a San Francesco ci sono state diverse tensioni. 
Quando è intervenuto il loro grado gerarchico di livello superiore ho chiesto che mettessero in 
iscritto i punti essenziali per continuare il cammino in San Francesco, ai quali avrei data risposta 
scritta. Sul tenore della mia risposta i responsabili del Cammino neocatecumenale hanno deciso di 
spostare le comunità altrove. Il Vescovo Lorenzo Bellomi non mi ha mai mosso alcun rimprovero 
sul mio atteggiamento fermo, tenuto nei confronti del cammino neocatecumenale; d'altra parte gli 
avevo detto, dopo la confusione creatasi, che avrei agito decisamente secondo la mia coscienza.
\bigbreak
\underline{Coloritura francescana della parrocchia}
\medbreak
Dopo la Missione al popolo, oltre che impegnarci nel mettere al centro la Parola di Dio nel 
cammino della comunità, abbiamo incominciato a riflettere sulla coloritura francescana che doveva 
avere la parrocchia di San Francesco: perché era sotto la protezione del santo patrono d'Italia,  
perché era affidata ad una comunità di frati francescani, perché al suo interno aveva già altre piccole 
presenze francescane come la GIFRA e l'OFS che potevano essere sviluppate. Così abbiamo 
improntato un cammino di sensibilizzazione francescana durato diversi mesi. Nel contesto di questo 
percorso francescano abbiamo partecipato come parrocchia al grande pellegrinaggio regionale in 
Assisi, il 4 ottobre, per l'offerta dell'olio per la lampada votiva; abbiamo iniziato un cammino di 
formazione specifica per entrare nell'OFS e infine il 28 ottobre '90 abbiamo ricordato il 25\textdegree\ della 
parrocchia, che ricorreva il 3 ottobre, quando eravamo in pellegrinaggio in Assisi. 
C'è stato un triduo di preparazione; in una serata il p. Provinciale, p. Agostino Gardin 
(attuale vescovo di Treviso), ha sviluppato il tema: “Caratteristiche di una parrocchia animata da 
frati francescani”. 
Domenica 28 ottobre con semplicità abbiamo ricordato i 25 anni di vita della parrocchia con la 
presenza del vescovo mons. Lorenzo Bellomi, dei due ex parroci viventi: p. Fortunato e p. Olindo e 
dei sacerdoti del decanato. Al termine della Messa solenne il Vescovo ha benedetta la rinnovata 
sede della Fraternitas e i nuovi impianti sportivi, completamente rinnovati grazie alla tenacia di p. 
Giambattista. 
Circa la coloritura francescana della comunità parrocchiale successivamente è stata eretta la 
Milizia dell’Immacolata (di p. Kolbe) grazie allo zelo mariano di p. Ruggero. 
\bigbreak
\underline{L'oratorio e l'impegno per i giovani}
\medbreak
Quando sono giunto in via Giulia, la struttura dell'oratorio era utilizzata prevalentemente per 
la catechesi e per gli incontri delle comunità neocatecumenali. P. Giambattista cercava spazi 
ricreativi. L'ho sempre sostenuto in questo legittimo obiettivo e così lui trovando vari appoggi ha 
provveduto a rinnovare gli infissi e l'arredo nelle varie sale e ha affrontati gli impegnativi lavori di 
sistemazione dei campi da gioco con i bagni esterni.
La sua inaspettata partenza nel novembre '93 ha creato un vuoto di presenza, con tante 
difficoltà nel portare avanti l'animazione della pastorale giovanile e la gestione dell'oratorio.
Nel settembre '94 è arrivato il novello diacono p. Enzo Piovesan, incaricato a inserirsi in quel 
settore: lo ha fatto con le sue caratteristiche e con i suoi convincimenti. Conosciamo le difficoltà, 
lavorando tra i giovani, di trovare il giusto dosaggio tra le esigenze di stare insieme divertendosi in 
molteplici maniere e la proposta religiosa esplicita per aiutarli a crescere nella fede e maturare una 
scelta gioiosa di Cristo Gesù.
\bigbreak
\underline{Conclusione}
\medbreak
Ho accennato ai momenti più significativi della vita della comunità parrocchiale, ma nella 
vita quotidiana ci sono stati tanti altri aspetti ben più importanti per me condensati nelle relazioni 
interpersonali con tante persone e con tante famiglie. In queste relazioni sincere e profonde, 
condividendo gioie e sofferenze con chi il Signore mi faceva incontrare sul cammino, ho 
sperimentato la gioia e la fecondità del mio ministero sacerdotale. 
\begin{flushright}
\textit{fra Lorenzo Gottardello}
\end{flushright}

%%========================================================

\end{document}
